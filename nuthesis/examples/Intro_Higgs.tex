\subsection{The Higgs Boson}
\label{Intro_Higgs}

% BAD SECTION
% NEEDS TO BE SIGNIFICANTLY REWORKED

\begin{figure}[htb]
  \begin{center}
    {\includegraphics[width=0.95\textwidth]{../figs/Intro/FeynmanHiggs.png}}
    \caption{Higgs production and decay}
    \label{fig:higgsProduction}
  \end{center}
\end{figure}

%  I feel that there is a number of imprecise statements here that need to be corrected. Here are examples.
%   (A side note: quant is not what you mean, check the dictionary. You want to say “quantum”)
%   I would not say that the Higgs boson is a quantum of the Higgs field. The Higgs field is a doublet of complex fields, i.e. it has four components. According to wikipedia wording, for example, “it is a quantum excitation of one of the four components of the Higgs field. "
% This part: "discovered by ATLAS and CMS collaborations in the reaction shown in Fig. 4 in γγ and ZZ decay channels”
% The diagram in Fig. 4 is not appropriate for ZZ.
% "the same approach can be used to introduced masses of all elementary particles.” - but what about neutrinos? I am not sure if our favorite explanation of neutrino masses is the Higgs.
% It is not intuitive to me that larger mass means stronger interaction with the Higgs field,
% I am not sure why you say that.
% "that is how is gets its inertia”: consider rephrasing.








