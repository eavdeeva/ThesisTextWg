\chapter{$W\gamma$ Cross Section Measurement}
\label{sec:AN_WgMeas}

The goal of the physics measurement of this dissertation is to measure the total and differential cross section of $pp \rightarrow l\nu\gamma + X$ in bins of the photon transverse momentum $P_T^\gamma$ using CMS data collected in~2012 at~$\sqrt{s}=8$~TeV center-of-mass collision energy. 

In the physics experiments we measure a number of events of the specific process, and determine the cross section in a certain kinematic phase space as

\begin{equation}
  \sigma = N/L,
\end{equation}
\noindent{where $N$ is the estimated number of events of the process of our interest in the given phase space, and $L$ is the luminosity of our collider. The measurement strategy can be represented as a chain of steps that is summarized in Tab.~\ref{tab:analysisOutline}. Chapters~\ref{sec:AN_Selection}-\ref{sec:AN_CrossSection} describe each step in details.}

\begin{table}[h]
  \small
  \begin{center}
  \caption{Measurement steps. The first column is the name of the step, the second and the third columns are algebraic representations of the steps for the differential and total cross section measurements respectively. }
  \begin{tabular}{|c|c|c|}
    \hline
    Step & $d\sigma/dP_{T}$ & $\sigma$ \\ \hline
    select events & $N_{sel}^j$ &    $N_{sel}$       \\ \hline
    subtract background & $N_{sign}^j = N_{sel}^j - N_{bkg}^j$ &    $N_{sign}=N_{sel}-N_{bkg}$       \\ \hline
    unfold   & $N_{acc}^i = U_{ij} \cdot N_{sign}^j$ &    $-$       \\ \hline
    correct for the acceptance and efficiency & $N_{true}^i = \frac{N_{acc}^i}{(A \times\epsilon)^i}$ &  $N_{true}=\frac{N_{sign}}{A\times\epsilon}$       \\ \hline
    divide over luminosity and bin width & $ \left( \frac{d\sigma}{dP_{T}^\gamma} \right) ^i = \frac{N_{true}^i}{L \cdot (\Delta P_T^\gamma)^i}$  &  $\sigma = N_{true}/L$       \\ \hline
    estimate systematic uncertainties &  &         \\ \hline
  \end{tabular}
  \label{tab:analysisOutline}
  \end{center}
\end{table}

%\subsection{Data sample}
The data sample we use in this analysis was recorded by the CMS experiment in~2012 in the LHC $pp$ collisions at~8 TeV. The data collected by single electron and single muon triggers are used as the signal samples while double muon and double electron datasets are used for data-driven background estimation. Only runs and luminosity sections certified by CMS are considered in the measurement, which means that good functioning of all CMS sub-detectors is required.

All simulation samples (often referred as Monte Carlo or MC samples) considered in this analysis are generated with MadGraph and reconstructed centrally by CMS simulation team. The simulated samples are reweighted to represent the distribution of the number of $pp$ interactions per bunch crossing (pileup or PU), as measured in the data. Information regarding signal and background simulated samples used for the analysis is given in Tab.~\ref{tab:mc_bkg_samples} alongside with the corresponding cross sections. The $Z$+jets process is often referred as Drell Yan + jets or DY+jets.

\begin{table}[h]
  \small
  \begin{center}
    \caption{Summary of simulated background samples used in the measurement.}
    \begin{tabular}{|l|l|l|}
      \hline
      Process                      & $\sigma$, pb         \\ \hline
      $W\gamma \rightarrow l\nu\gamma$     & 553.92 (NLO)    \\ \hline
      $W$+jets$ \rightarrow l\nu + jets$          & 36257.2 (NNLO)  \\  \hline
      $Z$+jets$ \rightarrow ll + jets$            & 3503.71         \\ \hline
      $t\bar{t}$+jets$\rightarrow 1$l+X         & 99.44 (NNLO)   \\ \hline
      $t\bar{t}$+jets$\rightarrow 2$l+X         & 23.83         \\ \hline
      $t\bar{t}\gamma$                    & 1.444          \\ \hline
      $Z\gamma \rightarrow ll\gamma$       & 171.62           \\ \hline
    \end{tabular}
    \label{tab:mc_bkg_samples}
  \end{center}
\end{table} 

The NLO cross section of $W\gamma$ was calculated with the MCFM in the same phase space for which the $W\gamma$ sample was generated. The NNLO contribution is estimated to be~19\%-26\% of the NLO value~\cite{ref_theory_NNLO}.

The cross section of $Z\gamma$ measured by CMS is taken~\cite{ref_Zg8TeV} and expanded to the phase space of the $Z\gamma$ generated MC sample. The resulting cross section is found to be~$171.62$~pb.

To avoid experimentalist's bias, the whole measurement was performed in a blinded way at first. Our blinded strategy was the following:
\begin{itemize}
  \item for $p_T^{\gamma}<45~GeV$: use full data; and
  \item for $p_T^{\gamma}>45~GeV$: use $5\%$ of data.
\end{itemize}
\noindent{After the whole measurement procedure was fully established, we unblinded the measurement. All plots shown in this dissertation are prepared with unblided data.}

The measurement was performed using advanced computing resources. A brief description of the software tools used and developed for the measurement is available in App.~\ref{sec:Code}.

