\section{Detector Resolution Unfolding}
\label{sec:Unfolding}

% EXPLAIN WHAT IS Detector Resolution UNFOLDING

%QUOTE
%The effect of detector resolution leads to a migration of events from bin i of the true
%invariant mass distribution to bin k of the reconstructed mass distribution. For a
%better comparison of observed dilepton spectra with theory, this effect of migration
%is corrected through unfolding. The procedure uses the yield distribution determined
%from simulation by mapping it onto the measured one to obtain the true distribu-
%tion. The unfolding procedures for differential and double-differential cross section
%calculations are described below.

%My words:
The finite detector resolution in measuring the energy of a photon causes bin-to-bin migration in the $P_T^{\gamma}$ spectrum. The reconstructed $P_T^{\gamma(reco)}$ may not coincide with the true $P_T^{\gamma(true)}$, and, therefore, the event reconstructed in a $P_T^{\gamma}$ bin $j$ may, in fact, belong to the bin $i \neq j$. To recover the true $P_T^{\gamma}$ spectrum, we apply the procedure of the detector resolution unfolding.

The reconstructed $P_T^{\gamma}$ spectrum is related to the true $P_T^{\gamma}$ spectrum as:
\begin{equation}\label{eq:response_matrix}
  N^{reco}_{j} = R_{ji} N^{A\times\epsilon}_{i},
\end{equation}
\noindent{where $N^{reco}_{j}$ and $N^{A\times\epsilon}_{i}$ are numbers of events in a given $P_T^{\gamma(reco)}$ and $P_T^{\gamma(true)}$ bins, respectively, $R_{ji}$ is the ``response matrix'' where each element is the probability of an event with true $P_T^{\gamma}$ in the bin ``$i$'' to be reconstructed with  $P_T^{\gamma}$ in the bin ``$j$''. The notation $N^{A\times\epsilon}_{i}$ is used because this yield is further corrected for the acceptance and efficiency (Ch.~\ref{sec:AccXEff}) and is consistent with the definition given in Tab.~\ref{tab:analysisOutline}.}

The simplest method to recover the true spectrum is to solve the system of linear equations Eq.~\ref{eq:response_matrix} if the $R_{ji}$ is known. However, this method often encounters numerical difficulties due to possible matrix singularity, large statistical fluctuations and the effect of oscillations of the unfolded spectrum. To avoid these difficulties, we use the D'Agostini method~\cite{ref_DAgostini} as recommended by the CMS SMP group. The D'Agostini method is based on the Bayes' theorem and unfolds the reconstructed spectrum iteratively.

%One possible way to recover the true  $P_T^{\gamma}$ spectrum is matrix inversion:}
%\begin{equation}
%  N^{A\times\epsilon}_i = \left( R_{ji} \right) ^{-1} N^{reco}_j.
%\end{equation}

The migration matrix $M_{ji}$ is prepared using the signal MC sample ($W\gamma\rightarrow\mu\nu_{\mu}\gamma$/$W\gamma\rightarrow{e}\nu_{e}\gamma$) where both true (gen-level) and reconstructed $P_T^\gamma$ spectra are known. The $M_{ji}$ contains the number of selected signal events in each [$j$,$i$] bin. 

After that, we pass the migration matrix $M_{ji}$, generated and reconstructed yields from signal MC $N^{gen-MC}_i$ and $N^{reco-MC}_j$ and reconstructed yields from data to the RooUnfold class~\cite{ref_RooUnfold} which performs unfolding using D'Agostini method with five iterations. Yields before and after detector resolution unfolding are compared in Tab.~\ref{tab:unf_results_MUON_WGamma} for the muon channel and in Tab.~\ref{tab:unf_results_ELECTRON_WGamma} for the electron channel. 

\begin{table}[h]
  \scriptsize
  \begin{center}
  \caption{$P_T^{\gamma}$ yields of $W\gamma$ before and after unfolding in the muon channel. Diagonal elements of the error matrix are shown as uncertainties for the unfolded yields.}
  \begin{tabular}{|c|c|c|}
\hline
  $P_T^{\gamma}$, &  \multicolumn{2}{|c|}{yields} \\ 
  GeV           & pre-unfolded &  unfolded  \\ \hline

 10 -  15 &     $39621 \pm 678$ &     $38843 \pm 718$  \\ \hline
 15 -  20 &     $19449 \pm 361$ &     $19662 \pm 438$  \\ \hline
 20 -  25 &     $11315 \pm 230$ &     $11443 \pm 275$  \\ \hline
 25 -  30 &     $8417 \pm 160$ &     $8714 \pm 196$  \\ \hline
 30 -  35 &     $5613 \pm 128$ &     $5529 \pm 155$  \\ \hline
 35 -  45 &     $7518 \pm 133$ &     $7895 \pm 149$  \\ \hline
 45 -  55 &     $2716 \pm  95$ &     $2605 \pm 108$  \\ \hline
 55 -  65 &     $2293 \pm  62$ &     $2307 \pm  70$  \\ \hline
 65 -  75 &     $1191 \pm  53$ &     $1198 \pm  62$  \\ \hline
 75 -  85 &     $1101 \pm  41$ &     $1165 \pm  48$  \\ \hline
 85 -  95 &     $757 \pm  33$ &     $776 \pm  41$  \\ \hline
 95 - 120 &     $1054 \pm  44$ &     $1064 \pm  50$  \\ \hline
120 - 500 &     $1107 \pm  39$ &     $1141 \pm  40$  \\ \hline
  \end{tabular}
  \label{tab:unf_results_MUON_WGamma}
  \end{center}
\end{table}

\begin{table}[h]
  \scriptsize
  \begin{center}
  \caption{$P_T^{\gamma}$ yields of $W\gamma$ before and after unfolding in the electron channel. Diagonal elements of the error matrix are shown as uncertainties for the unfolded yields.}
  \begin{tabular}{|c|c|c|}
\hline
  $P_T^{\gamma}$, &    \multicolumn{2}{|c|}{yields}  \\ 
  GeV           &  pre-unfolded &  unfolded  \\ \hline

 10 -  15 &     $9209\pm 378$ &     $9192\pm 413$  \\ \hline
 15 -  20 &     $4920\pm 319$ &     $4850\pm 380$  \\ \hline
 20 -  25 &     $3660\pm 212$ &     $3698\pm 249$  \\ \hline
 25 -  30 &     $2734\pm 127$ &     $2948\pm 159$  \\ \hline
 30 -  35 &     $2015\pm  84$ &     $2075\pm 102$  \\ \hline
 35 -  45 &     $2677\pm  83$ &     $2770\pm  91$  \\ \hline
 45 -  55 &     $1152\pm  81$ &     $1116\pm  93$  \\ \hline
 55 -  65 &     $1244\pm  70$ &     $1214\pm  80$  \\ \hline
 65 -  75 &     $881\pm  56$ &     $911\pm  63$  \\ \hline
 75 -  85 &     $579\pm  42$ &     $590\pm  48$  \\ \hline
 85 -  95 &     $483\pm  38$ &     $490\pm  45$  \\ \hline
 95 - 120 &     $664\pm  46$ &     $692\pm  50$  \\ \hline
120 - 500 &     $1020\pm  40$ &     $1052\pm  40$  \\ \hline
  \end{tabular}
  \label{tab:unf_results_ELECTRON_WGamma}
  \end{center}
\end{table}

After $P_T^{\gamma}$ spectrum is unfolded, measurements in different $P_T^{\gamma}$ bins become correlated. Correlation matrices are shown in Fig.~\ref{fig:corrMatrices_Wg}. 

For illustration purpose, in addition to the migration matrix we also prepare the response matrix $R_{ji}$ (Fig.~\ref{fig:respMatrices_Wg}) by normalizing the migration matrix in each $j$ bin to all events reconstructed in this bin. The response matrix is shown in Fig.~\ref{fig:respMatrices_Wg}.

To validate the procedure of the detector resolution unfolding, we perform the MC closure checks. Gen-level and reconstructed yields are prepared using the signal MC. Then reconstructed yields are smeared by the Gaussian distribution according to the statistical uncertainties on the yields. The smeared yields are unfolded and compared to the gen-level yields. In addition to the D'Agostini method, we check the performance of the matrix inversion method for the unfolding which recovers the true yields as $N^{A\times\epsilon}_i = (R_{ji})^{-1} N^{reco}_j$. 

The results of the MC closure checks are summarized in Tab.~\ref{tab:unf_mc_closure_MUON_WGamma}-\ref{tab:unf_mc_closure_ELECTRON_WGamma} for the muon and electron channels respectively. The unfolded yields show reasonable agreement to the gen-level yields except for the underflow bin ($10-15$~GeV). The disagreement in the underflow bin may be caused by migration between $P_T^{\gamma}<$10~GeV and 10$<P_T^{\gamma}<$15~GeV ranges because events with $P_T^{\gamma}<$10~GeV are not present in the signal MC samples.  

%$M^{corr}_{ij} = \frac{C_i \cdot C_j}{\sqrt{(C_{ii} \cdot C_{jj})}} $, where $C_{ij}$ is a matrix element of a covariance matrix.

%\begin{figure}[htb]
%  \begin{center}
%   \includegraphics[width=0.90\textwidth]{../figs/figs_v11/MUON_WGamma/Constants/cMigrMatrix_MUON_WGamma__yield_pm_stat.pdf}\\
%\includegraphics[width=0.90\textwidth]{../figs/figs_v11/ELECTRON_WGamma/Constants/cMigrMatrix_ELECTRON_WGamma__yield_pm_stat.pdf}
%  \caption{Migration matrix derived from the signal MC.}
%  \label{fig:migrMatrices_Wg}
%  \end{center}
%\end{figure}

\begin{figure}[htb]
  \begin{center}
   \includegraphics[width=0.90\textwidth]{../figs/figs_v11/MUON_WGamma/Constants/matrCorrelation_yield_pm_stat.pdf}\\
\includegraphics[width=0.90\textwidth]{../figs/figs_v11/ELECTRON_WGamma/Constants/matrCorrelation_yield_pm_stat.pdf}
  \caption{Correlation matrices of statistical uncertainties on unfolded $W\gamma$ yields in the muon (top) and electron (bottom) channels.}
  \label{fig:corrMatrices_Wg}
  \end{center}
\end{figure}

\begin{figure}[htb]
  \begin{center}
   \includegraphics[width=0.90\textwidth]{../figs/figs_v11/MUON_WGamma/Constants/cResponseMatr_MUON_WGamma__yield_pm_stat.pdf}\\
\includegraphics[width=0.90\textwidth]{../figs/figs_v11/ELECTRON_WGamma/Constants/cResponseMatr_ELECTRON_WGamma__yield_pm_stat.pdf}
  \caption{Response matrix derived from the signal MC in the muon (top) and electron (bottom) channels.}
  \label{fig:respMatrices_Wg}
  \end{center}
\end{figure}

%[1] https://indico.cern.ch/event/322577/ 
%[2] https://twiki.cern.ch/twiki/bin/view/CMS/TwikiSMP-GENRecommendations\#Unfolding\_How\_to

% WHAT IS MC CLOSURE CHECK

\begin{table}[h]
  \scriptsize
  \begin{center}
  \caption{Results of the MC closure test of the detector resolution unfolding of $P_T^{\gamma}$ yields of $W\gamma$ in the muon channel. The unfolding procedure is applied on the ``MC truth yields'', and the results of the matrix inversion (``inversion'') and D'Agostini (``D\'Agostini'') unfolding methods are compared to ``reconstructed yields'' and to each other.}
  \begin{tabular}{|c|c|c|c|c|}
  \hline
  $P_T^{\gamma}$,&  MC truth         &   reconstructed &  \multicolumn{2}{|c|}{unfolded yields} \\ 
  GeV          &  yields       &       yields        &  inversion &  D'Agostini \\ \hline
 10 -  15 &     $33888\pm 273$ &     $37074\pm 286$ &     $36226\pm206$ &     $36222\pm204$ \\ \hline
 15 -  20 &     $19736\pm 207$ &     $19181\pm 203$ &     $19612\pm171$ &     $19619\pm169$ \\ \hline
 20 -  25 &     $10364\pm 149$ &     $10171\pm 148$ &     $10358\pm122$ &     $10354\pm119$ \\ \hline
 25 -  30 &     $6254\pm 116$ &     $6156\pm 115$ &     $6233\pm96$ &     $6234\pm96$ \\ \hline
 30 -  35 &     $4026\pm  93$ &     $4007\pm  93$ &     $4010\pm81$ &     $4010\pm78$ \\ \hline
 35 -  45 &     $4516\pm  99$ &     $4461\pm  98$ &     $4502\pm79$ &     $4502\pm79$ \\ \hline
 45 -  55 &     $2731\pm  77$ &     $2680\pm  76$ &     $2724\pm57$ &     $2724\pm60$ \\ \hline
 55 -  65 &     $1662\pm  60$ &     $1686\pm  61$ &     $1655\pm45$ &     $1655\pm46$ \\ \hline
 65 -  75 &     $987\pm  46$ &     $945\pm  45$ &     $979\pm38$ &     $979\pm35$ \\ \hline
 75 -  85 &     $659\pm  38$ &     $638\pm  37$ &     $654\pm30$ &     $653\pm30$ \\ \hline
 85 -  95 &     $495\pm  33$ &     $480\pm  32$ &     $489\pm27$ &     $489\pm25$ \\ \hline
 95 - 120 &     $664\pm  38$ &     $663\pm  38$ &     $661\pm28$ &     $661\pm28$ \\ \hline
120 - 500 &     $726\pm  40$ &     $704\pm  39$ &     $720\pm26$ &     $720\pm27$ \\ \hline
500 - 2000 &     $2\pm   2$ &     $2\pm   2$ &     $2\pm1$ &     $2\pm1$ \\ \hline
  \end{tabular}
  \label{tab:unf_mc_closure_MUON_WGamma}
  \end{center}
\end{table}

\begin{table}[h]
  \scriptsize
  \begin{center}
  \caption{Results of the MC closure test of the detector resolution unfolding of $P_T^{\gamma}$ yields of $W\gamma$ in the electron channel. The unfolding procedure is applied on the ``MC truth yields'', and the results of the matrix inversion (``inversion'') and D'Agostini (``D\'Agostini'') unfolding methods are compared to ``reconstructed yields'' and to each other. }
  \begin{tabular}{|c|c|c|c|c|}
  \hline
  $P_T^{\gamma}$, &  MC truth         &   reconstructed          &  \multicolumn{2}{|c|}{unfolded yields} \\ \hline
  GeV &  yields       &       yields        &  inversion &  D'Agostini \\ \hline
 10 -  15 &     $16025\pm 185$ &     $16849\pm 190$ &     $17117\pm143$ &     $17116\pm141$ \\ 
 15 -  20 &     $8246\pm 131$ &     $8111\pm 130$ &     $8194\pm109$ &     $8196\pm108$ \\ \hline
 20 -  25 &     $4093\pm  92$ &     $4046\pm  92$ &     $4083\pm75$ &     $4082\pm74$ \\ \hline
 25 -  30 &     $2080\pm  66$ &     $1987\pm  64$ &     $2072\pm55$ &     $2072\pm55$ \\ \hline
 30 -  35 &     $1387\pm  54$ &     $1361\pm  54$ &     $1378\pm47$ &     $1378\pm46$ \\ \hline
 35 -  45 &     $1925\pm  64$ &     $1886\pm  63$ &     $1915\pm51$ &     $1915\pm50$ \\ \hline
 45 -  55 &     $1124\pm  49$ &     $1108\pm  48$ &     $1116\pm37$ &     $1116\pm38$ \\ \hline
 55 -  65 &     $855\pm  42$ &     $892\pm  43$ &     $848\pm33$ &     $848\pm34$ \\ \hline
 65 -  75 &     $655\pm  38$ &     $635\pm  37$ &     $649\pm30$ &     $649\pm28$ \\ \hline
 75 -  85 &     $447\pm  32$ &     $433\pm  32$ &     $442\pm24$ &     $442\pm24$ \\ \hline
 85 -  95 &     $316\pm  27$ &     $316\pm  27$ &     $311\pm21$ &     $311\pm20$ \\ \hline
 95 - 120 &     $507\pm  34$ &     $484\pm  33$ &     $501\pm23$ &     $501\pm23$ \\ \hline
120 - 500 &     $593\pm  37$ &     $575\pm  36$ &     $587\pm23$ &     $587\pm24$ \\ \hline
500 - 2000 &     $4\pm   3$ &     $4\pm   3$ &     $4\pm2$ &     $4\pm2$ \\ \hline
  \end{tabular}
  \label{tab:unf_mc_closure_ELECTRON_WGamma}
  \end{center}
\end{table}

