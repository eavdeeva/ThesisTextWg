\section{Systematic Uncertainties}
\label{sec:Systematics}

Each step of the measurement has uncertainties associated with this step. Each uncertainty is estimated as an uncertainty on yields, and is propagated through the further measurement steps to be converted into the uncertainty on the cross section. 

Uncertainties related to the subtraction of various background are estimated as uncertainties on ``data-bkg. yields''. To estimate uncerntainties of the effects listed above on the differential cross section, the uncertainties on yields are propagated through the unfolding, $A \times \epsilon$ correction, divided by the luminosity and by the bin width. For the total cross section, the uncertainties on ``data-bkg.'' yields are divided by the $A \times \epsilon$ correction and luminosity. 

When an uncertainty is propagated through the unfolding, the yields in the different $P_T^{\gamma}$ become correlated. The correlation matrices on the unfolded yields corresponding to all each uncertainty related to the background subtraction are provided in App.~\ref{sec:corrMatrices}, as well as the correlation matrix related to the unfolding procedure itself. Uncertainties related to post-unfolding steps of the $W\gamma$ measurement do not have to be propagated through unfolding and, thus, do not have corresponding correlation matrices. 

Uncertainties related to jets$\rightarrow\gamma$ background estimation are described in Ch.~\ref{sec:Systematics_jtog} while uncertainties related to the other measurement steps are described in Ch.~\ref{sec:Systematics_OtherSources}. Ch.~\ref{sec:Systematics_summary} summarizes relative systematic uncertainties originating from different sources.

\subsection{Uncertainties Related to Jets$\rightarrow\gamma$ Background Estimation}
\label{sec:Systematics_jtog}

The selected data samples in both muon and electron channels are composed mostly of the jets$\rightarrow\gamma$ events. Because there are more background than signal events in our selected sampls, any uncertainty on the background yields transforms to larger relative uncertainty of the signal yields. The uncertainties related to jets$\rightarrow\gamma$ background estimation are dominant sources of uncertainties in all $P_T^{\gamma}$ bins in the muon channels and in bins with $P_T^{\gamma}<$55~GeV in the electron channel.

The following sources contribute to the uncertainty of  the jets$\rightarrow\gamma$ background estimation:
\begin{itemize}
  \item the biases in the template shapes and the fit procedure;
  \item the uncertainty on the normalizations of $Z\gamma$ and DY+jets MC samples when the real-$\gamma$ (fake-$\gamma$) portions are subtracted from the ISR (FSR) templates. 
  \item the limited statistical power of the fake-$\gamma$ and real-$\gamma$ templates.
\end{itemize}

The systematic uncertainty on ``data-bkg.'' yields due to the bias in the template shapes and the fit procedure is computed as the difference between fit results of $I_{ch}^\gamma$ and $\sigma_{i\eta i\eta}^\gamma$ distributions
\begin{equation}
  \Delta N_{Ich\_vs\_\sigma i\eta i\eta} = |N_{Ich}-N_{\sigma i\eta i\eta}|,
\end{equation}
\noindent{where $N_{Ich}$ and $N_{\sigma i\eta i\eta}$ are signal yields obtained with fits of $I_{ch}^{\gamma}$ and $\sigma_{i\eta i\eta}$, respectively. These two variables are chosen because they are fairly independent: $I_{ch}^{\gamma}$ indicates charged particle activity around the photon, while $\sigma_{i\eta i\eta}^{\gamma}$ describes the shape of the shower induced by this photon in the ECal. If the template shapes were correct representations of the real-$\gamma$ and fake-$\gamma$ $I_{ch}^{\gamma}$ and $\sigma_{i\eta i\eta}$ distributions in data, and fits always resulted in a correct numbers of real-$\gamma$ and fake-$\gamma$ events, then the results of fits of these two variables would be consistent. However, the two sets of fit results are found to dramatically disagree. The difference between the results $|N_{Ich}-N_{\sigma i\eta i\eta}|$ is assigned as a source of the systematic uncertainty on ``data-bkg.'' yields.}

The uncertainty related to the limited statistical power of the data which are used to prepare templates is computed by separately randomizing the real-$\gamma$ and the fake-$\gamma$ templates. We prepare 20 real-$\gamma$ and 100 fake-$\gamma$ templates by randomizing our nominal templates with the Gaussian distribution. Then we perform fits with new templates and take the standard deviation of the fit results as an uncertainty. The uncertainties are computed separately for the real-$\gamma$ and the fake-$\gamma$ templates. The statistical uncertainty of the fake-$\gamma$ template is larger.

The results of the systematic uncertainty of $|N_{Ich}-N_{\sigma i\eta i\eta}|$ and the template statistical uncertainty are summarized in the Tab.~\ref{tab:diff_ways_to_fit_phoEt_muon}-\ref{tab:diff_ways_to_fit_phoEt_electron}. The column ``yield data-bkg.'' is the background subtracted yield which is used for the cross section measurement. The central values of these yields are taken from the ``data $I_{ch}^{\gamma}$'' column. The uncertainties in column ``sig. MC ($W\gamma\rightarrow[\mu/e]\nu\gamma$)''are statistical uncertainties of the signal MC samples. The uncertainties in columns ``data $I_{ch}^{\gamma}$'', ``data $\sigma_{i\eta i\eta}^\gamma$'', ``MC closure $I_{ch}^{\gamma}$'', ``MC closure $\sigma_{i\eta i\eta}^\gamma$'' include statistical uncertainties and systematic uncertainties originated from the limited statistical power of [pseudo]data used to prepare templates for jets$\rightarrow\gamma$ background estimation. Two uncertainties in the column ``yield data-bkg.'' are uncertainties estimated as $|N_{Ich}-N_{\sigma i\eta i\eta}|$ and uncertainties originated from the limited statistical power of data used to prepare real-$\gamma$ and fake-$\gamma$ templates. The values are shown in a format $N\pm\Delta N(I_{ch} vs \sigma_{i\eta i\eta} templ.)\pm \Delta N(templ. stat.)$ to compare these two uncertainties side-by-side.  %The uncertainties include the statistical uncertainties and the uncertainties due to template statistics.

\begin{table}[h]
  \tiny
  \begin{center}
  \caption{The $P_T^{\gamma}$ yields of $W\gamma\rightarrow\mu\nu\gamma$ data and pseudodata after the full background subtraction with jets$\rightarrow\gamma$ background subtracted based on fits of $I_{ch}^{\gamma}$ and  $\sigma_{i\eta i\eta}^\gamma$ distributions. The ``yields data-bkg.'' are the background subtracted yields that are passed to the further measurement steps. The central values for these yields coincide with ``data $I_{ch}^{\gamma}$'', the first uncertainties are estimated as $|N_{Ich}-N_{\sigma i\eta i\eta}|$, and the second uncertainties are uncertaintries related to the limits of the statistical power of data samples used to prepare $I_{ch}^{\gamma}$ templates. The signal MC yields ``sig. MC ($W\gamma\rightarrow\mu\nu\gamma$)'' are provided for the comparison purpose.}
%Results of the background subtraction based on fits of different variables. $W\gamma$, muon channel. First column: $P_T^{\gamma}$ ranges, second column: signal MC prediction, third and fourth columns: signal yields extracted from fits of $I_{ch}^{\gamma}$ and $\sigma_{i\eta i\eta}^\gamma$ distributions of data, fifth and sixth columns: pseudo signal yields extracted from MC closure tests, seventh column: signal yield that is used in the next measurement steps.
  \begin{tabular}{|c|c|c|c|c|c|c|}
    $P_T^{\gamma}$, &  sig. MC   & \multicolumn{2}{|c|}{data}  & \multicolumn{2}{|c|}{pseudodata} & yields\\ 
    GeV & ($W\gamma\rightarrow\mu\nu\gamma$) & $I_{ch}^{\gamma}$ & $\sigma_{i\eta i\eta}^\gamma$  & $I_{ch}^{\gamma}$  & $\sigma_{i\eta i\eta}^\gamma$   & data-bkg. \\ \hline
    \multicolumn{7}{|c|}{barrel photons} \\ \hline
    10-15 & 26250$\pm$240 & 30779$\pm$1919 & 26866$\pm$3134 & 29753$\pm$2476 & 35169$\pm$3726 &30779$\pm$3913$\pm$1865  \\ \hline
    15-20 & 12706$\pm$164 & 14620$\pm$1070 & 19641$\pm$1771 & 10809$\pm$1079 & 14990$\pm$2123 &14620$\pm$5021$\pm$1041  \\ \hline
    20-25 & 6793$\pm$120 & 8412$\pm$711 & 10446$\pm$4313 & 7746$\pm$626 & 9447$\pm$1741 &8412$\pm$2033$\pm$693  \\ \hline
    25-30 & 4087$\pm$93 & 5543$\pm$685 & 7179$\pm$3437 & 4459$\pm$532 & 5061$\pm$2094 &5543$\pm$1636$\pm$675  \\ \hline
    30-35 & 2603$\pm$74 & 3438$\pm$500 & 5181$\pm$2581 & 3451$\pm$197 & 3296$\pm$1156 &3438$\pm$1742$\pm$490  \\ \hline
    35-45 & 2971$\pm$80 & 5033$\pm$466 & 5587$\pm$3366 & 4515$\pm$308 & 4394$\pm$1632 &5033$\pm$554$\pm$454  \\ \hline
    45-55 & 1861$\pm$63 & 1458$\pm$410 & 2923$\pm$593 & 1828$\pm$277 & 2493$\pm$146 &1458$\pm$1464$\pm$402  \\ \hline
    55-65 & 1135$\pm$49 & 1626$\pm$207 & 1693$\pm$501 & 1165$\pm$214 & 1497$\pm$311 &1626$\pm$67$\pm$201  \\ \hline
    65-75 & 664$\pm$37 & 881$\pm$43 & 1105$\pm$271 & 787$\pm$193 & 694$\pm$162 &881$\pm$223$\pm$7  \\ \hline
    75-85 & 451$\pm$31 & 720$\pm$34 & 772$\pm$88 & 631$\pm$106 & 711$\pm$143 &720$\pm$52$\pm$0  \\ \hline
    85-95 & 340$\pm$27 & 511$\pm$139 & 510$\pm$175 & 464$\pm$62 & 451$\pm$98 &511$\pm$0$\pm$136  \\ \hline
    95-120 & 453$\pm$31 & 658$\pm$105 & 749$\pm$31 & 730$\pm$113 & 593$\pm$83 &658$\pm$91$\pm$98  \\ \hline
    120-500 & 546$\pm$34 & 842$\pm$214 & 824$\pm$63 & 809$\pm$105 & 710$\pm$191 &842$\pm$18$\pm$211  \\ \hline
    \multicolumn{7}{|c|}{endcap photons} \\ \hline
     10-15 & 10823$\pm$154 & 8840$\pm$2242 & 7936$\pm$2947 & 16556$\pm$2900 & -2631$\pm$1967 &8840$\pm$903$\pm$2184  \\ \hline
    15-20 & 6474$\pm$119 & 4829$\pm$1132 & 5089$\pm$1518 & 6490$\pm$1142 & 2686$\pm$2124 &4829$\pm$260$\pm$1101  \\ \hline
    20-25 & 3377$\pm$86 & 2902$\pm$729 & 4842$\pm$1329 & 5418$\pm$578 & 4483$\pm$1291 &2902$\pm$1939$\pm$710  \\ \hline
    25-30 & 2068$\pm$67 & 2873$\pm$408 & 3460$\pm$1514 & 3154$\pm$356 & 3920$\pm$975 &2873$\pm$586$\pm$394  \\ \hline
    30-35 & 1403$\pm$55 & 2174$\pm$306 & 1699$\pm$693 & 1821$\pm$401 & 1495$\pm$545 &2174$\pm$474$\pm$295  \\ \hline
    35-45 & 1489$\pm$57 & 2485$\pm$339 & 2956$\pm$1009 & 2405$\pm$279 & 2204$\pm$935 &2485$\pm$471$\pm$329  \\ \hline
    45-55 & 818$\pm$42 & 1257$\pm$243 & 1196$\pm$595 & 905$\pm$176 & 1150$\pm$226 &1257$\pm$61$\pm$237  \\ \hline
    55-65 & 550$\pm$34 & 666$\pm$208 & 966$\pm$375 & 581$\pm$219 & 329$\pm$260 &666$\pm$299$\pm$204  \\ \hline
    65-75 & 280$\pm$24 & 308$\pm$169 & 604$\pm$206 & 476$\pm$80 & 457$\pm$141 &308$\pm$295$\pm$166  \\ \hline
    75-85 & 186$\pm$20 & 380$\pm$162 & 378$\pm$91 & 249$\pm$69 & 200$\pm$66 &380$\pm$1$\pm$161  \\ \hline
    85-95 & 139$\pm$17 & 245$\pm$60 & 254$\pm$28 & 322$\pm$19 & 203$\pm$30 &245$\pm$8$\pm$57  \\ \hline
    95-120 & 208$\pm$21 & 395$\pm$55 & 374$\pm$195 & 353$\pm$41 & 173$\pm$54 &395$\pm$21$\pm$51  \\ \hline
    120-500 & 157$\pm$18 & 263$\pm$88 & 302$\pm$17 & 265$\pm$53 & 189$\pm$30 &263$\pm$38$\pm$85  \\ \hline
  \end{tabular}
  \label{tab:diff_ways_to_fit_phoEt_muon}
  \end{center}
\end{table}

\begin{table}[h]
  \tiny
  \begin{center}
  \caption{The $P_T^{\gamma}$ yields of $W\gamma\rightarrow e\nu\gamma$ data and pseudodata after the full background subtraction with jets$\rightarrow\gamma$ background subtracted based on fits of $I_{ch}^{\gamma}$ and  $\sigma_{i\eta i\eta}^\gamma$ distributions. The ``yields data-bkg.'' are the background subtracted yields that are passed to the further measurement steps. The central values for these yields coincide with ``data $I_{ch}^{\gamma}$'', the first uncertainties are estimated as $|N_{Ich}-N_{\sigma i\eta i\eta}|$, and the second uncertainties are uncertaintries related to the limits of the statistical power of data samples used to prepare $I_{ch}^{\gamma}$ templates. The signal MC yields ``sig. MC ($W\gamma\rightarrow e\nu\gamma$)'' are provided for the comparison purpose.}
  \begin{tabular}{|c|c|c|c|c|c|c|}
    $P_T^{\gamma}$, &  sig. MC   & \multicolumn{2}{|c|}{data}  & \multicolumn{2}{|c|}{MC closure} & yield\\ 
    GeV & ($W\gamma\rightarrow e\nu\gamma$) & $I_{ch}^{\gamma}$ & $\sigma_{i\eta i\eta}^\gamma$  & $I_{ch}^{\gamma}$  & $\sigma_{i\eta i\eta}^\gamma$   & data-bkg. \\ \hline
    \multicolumn{7}{|c|}{barrel photons} \\ \hline
    10-15 & 12480$\pm$163 & 10994$\pm$1331 & 12425$\pm$2000 & 10640$\pm$1500 & 14995$\pm$2225 &10994$\pm$1430$\pm$1277  \\ \hline
    15-20 & 5857$\pm$110 & 5160$\pm$668 & 7421$\pm$1173 & 4124$\pm$602 & 5721$\pm$1927 &5160$\pm$2261$\pm$613  \\ \hline
    20-25 & 2868$\pm$77 & 3022$\pm$384 & 3168$\pm$2937 & 3390$\pm$258 & 3699$\pm$1261 &3022$\pm$145$\pm$338  \\ \hline
    25-30 & 1411$\pm$54 & 1846$\pm$293 & 2250$\pm$1984 & 1365$\pm$152 & 1339$\pm$1167 &1846$\pm$404$\pm$273  \\ \hline
    30-35 & 915$\pm$43 & 1283$\pm$193 & 1831$\pm$971 & 877$\pm$111 & 891$\pm$278 &1283$\pm$547$\pm$180  \\ \hline
    35-45 & 1247$\pm$51 & 1732$\pm$190 & 1965$\pm$882 & 1359$\pm$111 & 1330$\pm$277 &1732$\pm$232$\pm$178  \\ \hline
    45-55 & 820$\pm$41 & 673$\pm$207 & 1199$\pm$485 & 698$\pm$118 & 933$\pm$65 &673$\pm$526$\pm$196  \\ \hline
    55-65 & 654$\pm$37 & 956$\pm$302 & 1010$\pm$157 & 566$\pm$95 & 666$\pm$152 &956$\pm$53$\pm$296  \\ \hline
    65-75 & 440$\pm$30 & 625$\pm$252 & 756$\pm$47 & 357$\pm$99 & 458$\pm$123 &625$\pm$131$\pm$248  \\ \hline
    75-85 & 295$\pm$25 & 367$\pm$137 & 516$\pm$134 & 339$\pm$45 & 285$\pm$84 &367$\pm$148$\pm$132  \\ \hline
    85-95 & 234$\pm$22 & 364$\pm$29 & 366$\pm$33 & 315$\pm$63 & 283$\pm$83 &364$\pm$1$\pm$2  \\ \hline
    95-120 & 318$\pm$26 & 430$\pm$88 & 555$\pm$66 & 397$\pm$77 & 400$\pm$135 &430$\pm$124$\pm$78  \\ \hline
    120-500 & 429$\pm$30 & 743$\pm$234 & 734$\pm$40 & 568$\pm$54 & 537$\pm$236 &743$\pm$9$\pm$231  \\ \hline
    \multicolumn{7}{|c|}{endcap photons} \\ \hline
    10-15 & 4368$\pm$96 & -1785$\pm$122 & 4129$\pm$1180 & 2286$\pm$1356 & -1502$\pm$1196 &-1785$\pm$5915$\pm$108  \\ \hline
    15-20 & 2253$\pm$68 & -241$\pm$537 & 1869$\pm$762 & 1541$\pm$483 & 352$\pm$759 &-241$\pm$2110$\pm$506  \\ \hline
    20-25 & 1177$\pm$49 & 637$\pm$298 & 1679$\pm$534 & 1308$\pm$192 & 1414$\pm$481 &637$\pm$1042$\pm$277  \\ \hline
    25-30 & 574$\pm$34 & 887$\pm$147 & 1078$\pm$646 & 674$\pm$117 & 1125$\pm$370 &887$\pm$190$\pm$131  \\ \hline
    30-35 & 445$\pm$31 & 731$\pm$107 & 555$\pm$249 & 451$\pm$119 & 355$\pm$155 &731$\pm$176$\pm$96  \\ \hline
    35-45 & 638$\pm$37 & 943$\pm$116 & 1071$\pm$326 & 773$\pm$76 & 789$\pm$189 &943$\pm$127$\pm$104  \\ \hline
    45-55 & 287$\pm$24 & 478$\pm$106 & 449$\pm$449 & 307$\pm$67 & 347$\pm$78 &478$\pm$28$\pm$95  \\ \hline
    55-65 & 237$\pm$22 & 287$\pm$155 & 433$\pm$44 & 225$\pm$51 & 220$\pm$114 &287$\pm$145$\pm$150  \\ \hline
    65-75 & 194$\pm$21 & 255$\pm$73 & 372$\pm$38 & 154$\pm$45 & 37$\pm$87 &255$\pm$116$\pm$67  \\ \hline
    75-85 & 137$\pm$18 & 210$\pm$47 & 236$\pm$28 & 201$\pm$59 & 155$\pm$73 &210$\pm$25$\pm$40  \\ \hline
    85-95 & 81$\pm$14 & 118$\pm$47 & 128$\pm$30 & 146$\pm$39 & 44$\pm$40 &118$\pm$10$\pm$40  \\ \hline
    95-120 & 166$\pm$20 & 233$\pm$51 & 211$\pm$21 & 224$\pm$21 & 192$\pm$49 &233$\pm$21$\pm$46  \\ \hline
    120-500 & 145$\pm$18 & 276$\pm$21 & 254$\pm$24 & 227$\pm$31 & 194$\pm$46 &276$\pm$22$\pm$3  \\ \hline
  \end{tabular}
  \label{tab:diff_ways_to_fit_phoEt_electron}
  \end{center}
\end{table}

Another source of the systematic uncertainty related to jets$\rightarrow\gamma$ background estimation $\Delta N_{Norm}$ originates from the  uncertainty on the $Z\gamma$ MC normalization. The $Z\gamma$ MC sample is used to prepare fake-$\gamma$ template, and the normalization of this sample significantly affects the template shape. In fact, uncertainty on DY+jets MC sample normalization also contributes to the uncertainty of the cross section because DY+jets MC sample is used to subtract fake-$\gamma$ contribution from FSR-selected $Z\gamma\rightarrow\mu\mu\gamma$ sample. These both contributions are accumulated in $\Delta \sigma^{Norm}$, however, the contribution from the uncertainty on the DY+jets MC sample normalization is very small compared to the contribution from the uncertainty on the $Z\gamma$ MC sample normalization.  

The uncertainty on the $Z\gamma$ normalization is set to be~4.6\% as reported by CMS $Z\gamma$ measurement at $\sqrt{s}$=8~TeV~\cite{ref_Zg8TeV}. To estimate $\Delta N_{Norm}$, we prepare templates with $Z\gamma$ normalizations deviated by $\pm$4.6\% from the nominal value. After that, we perform fits with such deviated templates, and compare results among the fits with templates of nominal normalization and with two deviated ones. The spread among three results is a systematic uncertainty on the ``data-bkg.'' yields. 

 Systematic uncertainties related to jets$\rightarrow\gamma$ background estimation are propagated through unfolding and other measurement steps. Resulting uncertainties on the cross section are listed among the major uncertainties in Tab.~\ref{tab:systInPercent_MUON_WGamma}-\ref{tab:systInPercent_ELECTRON_WGamma} in columns ``syst $|N_{Ich}-N_{\sigma{i\eta i\eta}}|$'', ``$Z\gamma$ MC norm'', and ``temp stat''.

\subsection{Other Sources of the Systematic Uncertainties}
\label{sec:Systematics_OtherSources}
% e to gamma

Another significant uncertainty only appears in the electron channel, it is the uncertainty related to $e\rightarrow\gamma$ background estimation. This uncertainty has components related to the fit bias and to the limitated statistical power of MC samples unvolved in this background estimation. 

To estimate the uncertainty due to fit bias, we perform fits of $Z$-peak on two data samples. One of them is prepared by applying all $W\gamma$ selection criteria except $Z$-mass window requirement, and the other one is prepared by applying all $W\gamma$ selection criteria except $Z$-mass window and $M_T^{\gamma}$ requirements. For the second case, we apply efficiency of $M_T^W$ selection requirement on the fit result. Whether $M_T^W$ selection requirement is applied or not, the data sample can be described by the same function, which must result in the amount of $e\rightarrow\gamma$ events in the nominally selected sample. The difference in the number of $e\rightarrow\gamma$ background events indicates a fit bias. The plots with the fit results of the datasets before and after $M_W^T$ requirement applied are shown in App.~\ref{sec:EtogammaFitPlotsNoWMtCut} and~\ref{sec:EtogammaFitPlots} respectively.

Another source of uncertainty originates from the limited statistical power of all MC samples involved in the $e\rightarrow\gamma$ background estimation. This uncertainty is taken care of by RooFit~\cite{ref_RooFit} which provides us with uncertainties on the $N_{e\rightarrow\gamma}$ determined by fit (Eq.~\ref{eq:fit_function_etog}) and by ROOT~\cite{ref_ROOT} which treats properly the uncertainties on weighted $M_{e\gamma}$ histograms involved in the algebraic expression~\ref{eq:Scale_etog}. Values of $e\rightarrow\gamma$ uncertainties from both sources are propagated through unfolding and other measurement steps and summarized in Tab.~\ref{tab:systInPercentEtogamma_ELECTRON_WGamma}.

% real gamma

For the real-$\gamma$ background subtraction, the statistical uncertainties of $Z\gamma$ and $W\gamma\rightarrow\tau\nu\gamma$ samples and their normalization uncertainties are taken into account. The normalization uncertainty applied for the $Z\gamma$ sample is~4.6\% as reported by CMS~8~TeV~$Z\gamma$ measurement and for the $W\gamma\rightarrow\tau\nu\gamma$ is~20\% because we use NLO value, and the NNLO contribution is estimated to have an order of~20\%. These uncertainties are minor.  They are propagated through unfolding and other measurement steps, and are listed in Tab.~\ref{tab:systInPercentSmallSysts_MUON_WGamma}-\ref{tab:systInPercentSmallSysts_ELECTRON_WGamma} in ``real-$\gamma$ bkg'' columns.

% unfolding and Acc X Eff

The migration matrix for the unfolding and $A\times\epsilon$ correction constants are derived from the signal MC sample. Limited statistical power of the signal MC sample contributes to the systematic uncertainty of the differential cross section through the unfolding procedure and to both differential and total cross section through the  $A\times\epsilon$ correction. 

To evaluate the uncertainty related to the limited signal MC statistical power for the migration matrix, first, we randomize the migration matrix 100 times by Gaussian distribution as $M_{ji}\rightarrow Gaus(M_{ji},\sigma_{ji})$ where $\sigma_{ji}$ is the signal MC statistical uncertainties in particular $[j,i]$ bin. After that, the procedure of unfolding is repeated for each migration matrix. The standard deviation out of all unfolding outputs is taken as an uncertainty on unfolded yields in each $P_T^{\gamma}$ bin, and, finally, the uncertainty is propagated through the $A\times\epsilon$ correction and is divided by the luminosity and the bin width to estimate the uncertainty on the cross section.

To evaluate the uncertainty related to the limited signal MC statistical power for the $A \times \epsilon$ correction constants, we use the expression
\begin{equation}
\Delta N_{true}^i= N_{A\times \epsilon}^i \cdot \frac{\Delta{(A\times \epsilon)^i}} { ((A\times \epsilon)^{i2})}, 
\end{equation}
\noindent{where $N_{A\times \epsilon}^i$ are unfolded yields, before $A \times \epsilon$ correction as defined in Tab.~\ref{tab:analysisOutline}. To estimate uncertainty on the cross section, $\Delta N_{true}^i$ is divided by the luminosity and by the bin width.  These uncertainties are minor, and are listed in Tab.~\ref{tab:systInPercentSmallSysts_MUON_WGamma}-\ref{tab:systInPercentSmallSysts_ELECTRON_WGamma} in ``unf. MC stat'' and ``$A \times \epsilon$ MC stat'' columns.}

% MET

Another source of the systematic uncertainty originates from biases in $E_T^{miss}$ modeling in the MC. These biases affect procedures of detector resolution unfolding and $A\times\epsilon$ correction. To estimate this uncertainty, we prepare two additional signal MC samples with $M_T^{W} \rightarrow M_T^W \pm \sigma^{\pm}_{MTW}$. To determine $\sigma^{\pm}_{MTW}$, we changes values of the $P_T$ of the photons, electrons (for the electron channel) and jets in the event by their uncertainties as prescribed by CMS EGamma and JetMET POG as $P_T^{\gamma}\rightarrow P_T^{\gamma} \pm \Delta P_T^{\gamma}$, $P_T^{e}\rightarrow P_T^{e} \pm \Delta P_T^{e}$, $P_T^{jets}\rightarrow P_T^{jets} \pm \Delta P_T^{jets}$. Then sum up all the listed contributions as the Lorentz vectors, and recalculate values of $E_T^{miss}$ and, therefore, of $M_T^{W}$. In these new MC samples we apply selection requirements on these alternative $M_T^W$ values, and, therefore, obtain new selected signal MC samples. Using these new samples, we compute $A \times \epsilon$ and prepare migration matrices. After that, we compute two additional cross section values based on new $A\times\epsilon$ values and migration matrices. The spread in the cross section among the three results, including the nominal one, is the systematic uncertainty. This uncertainty is minor, and the values are provided in Tab.~\ref{tab:systInPercentSmallSysts_MUON_WGamma}-\ref{tab:systInPercentSmallSysts_ELECTRON_WGamma} in ``$M_T^W$ req.'' column.

% SF

The contribution from the uncertainties of the efficiency SFs are also estimated. The SFs are varied by $\pm 1\sigma$, then the new $A \times \epsilon$ values and migration matrices are obtained, and new values of the cross section are found. The spread in the cross section among the three results, with $+1\sigma$, $-1\sigma$ and the nominal scale factor values, is the systematic uncertainty. The contribution of the SF systematic uncertainty in the muon channel is minor, however, in the electron channel it is very significant and in certain $P_T^{\gamma}$ bins is even dominant (Tab.~\ref{tab:systInPercent_MUON_WGamma}-\ref{tab:systInPercent_ELECTRON_WGamma}). The SF uncertainty in the electron channel is so large because we required PSV instead of CSEV to select $W\gamma\rightarrow e\nu\gamma$ events, therefore, we could not use SFs provided by EGamma POG but had to use the SFs provided by $W\gamma\gamma$ measurement team instead. Those SFs were prepared using a very small data sample resulting in large uncertainties of SFs which convert into large uncertainties of the $W\gamma\rightarrow e\nu\gamma$ cross section. 

%TODO:\\
%Momentum scale and resolution (variations prescribed by POGs). Need to think more carefully how to implement technically \\

% PU, Lumi

The systematic uncertainty related to PU reweighting is estimated by varying the PU cross section by $\pm$5\%. Similarly to the uncertainties related to $E_T^{miss}$ and SFs, we prepare two additional signal MC samples with alternative values of PU weight, prepare new $A \times \epsilon$ constants and migration matrices, and compute new cross section values. The spread in the cross section among the three results corresponding to the nominal PU cross section and to those changed by $\pm$5\% is the systematic uncertainty. This uncertainty is minor, and the values are provided in Tab.~\ref{tab:systInPercentSmallSysts_MUON_WGamma}-\ref{tab:systInPercentSmallSysts_ELECTRON_WGamma} in the ``PU weight'' column.

The luminosity uncertainty is~2.6\% which converts to~2.6\% uncertainty of cross section in all $P_T^{\gamma}$ bins. This systematic uncertainty is listed among the major uncertainties in Tab.~\ref{tab:systInPercent_MUON_WGamma}-\ref{tab:systInPercent_ELECTRON_WGamma} in the ``syst lumi'' column.

\subsection{Summary of the Systematic Uncertainties}
\label{sec:Systematics_summary}

The relative systematic uncertainties are summarized in Tab.~\ref{tab:systInPercent_MUON_WGamma} and Tab.~\ref{tab:systInPercent_ELECTRON_WGamma} for the muon and electron channels respectively. The systematic uncertainties related to pre-unfolding measurement steps have to be propagated through unfolding. For each of such uncertainties, a correlation matrix appears. All these correlation matrices are plotted in App.~\ref{sec:corrMatrices}.

\begin{table}[h]
  \scriptsize
  \begin{center}
  \caption{Relative uncertainties [\%] on the $W\gamma$ differential cross section in the muon channel. The details of the ``other'' column are provided in Tab.~\ref{tab:systInPercentSmallSysts_MUON_WGamma}. The `` total'' is the total relative systematic uncertainty on $d\sigma/dP_T^{\gamma}$.}
   \begin{tabular}{|c|c|c|c|c|c|c|c|c|}
                   &     & \multicolumn{7}{|c|}{systematic errors}     \\
    $P_T^{\gamma}$,  & stat & \multicolumn{3}{|c|}{related to jets$\rightarrow\gamma$}              &  &  &  & \\
    GeV           & err & $|N_{Ich}-N_{\sigma{i\eta i\eta}}|$ &$Z\gamma$ MC norm         &templ. stat  & SFs & lumi & other & total\\ \hline
    total  & 1 & 10 & 24 & 4 & 2 & 3 & 4 & 27 \\ \hline
%    10-15 & 2 & 11 & 7 & 8 & 0 & 3 & 6 & 16 \\ \hline
    15-20 & 2 & 31 & 12 & 10 & 3 & 3 & 6 & 35 \\ \hline
    20-25 & 2 & 29 & 13 & 11 & 1 & 3 & 6 & 34 \\ \hline
    25-30 & 2 & 24 & 13 & 11 & 1 & 3 & 5 & 30 \\ \hline
    30-35 & 3 & 40 & 15 & 13 & 2 & 3 & 7 & 45 \\ \hline
    35-45 & 2 & 11 & 12 & 8 & 2 & 3 & 6 & 19 \\ \hline
    45-55 & 4 & 62 & 19 & 20 & 2 & 3 & 8 & 68 \\ \hline
    55-65 & 3 & 15 & 12 & 14 & 1 & 3 & 7 & 24 \\ \hline
    65-75 & 6 & 36 & 19 & 17 & 1 & 3 & 10 & 44 \\ \hline
    75-85 & 4 & 6 & 11 & 16 & 1 & 3 & 10 & 21 \\ \hline
    85-95 & 5 & 2 & 9 & 23 & 1 & 3 & 13 & 25 \\ \hline
    95-120 & 5 & 10 & 8 & 12 & 1 & 3 & 9 & 18 \\ \hline
    120-500 & 3 & 4 & 11 & 21 & 2 & 3 & 9 & 24 \\ \hline
  \end{tabular}
  \label{tab:systInPercent_MUON_WGamma}
  \end{center}
\end{table}

\begin{table}[h]
  \scriptsize
  \begin{center}
  \caption{Relative uncertainties [\%] on the $W\gamma$ differential cross section in the electron channel. The details of the ``other'' column are provided in Tab.~\ref{tab:systInPercentSmallSysts_MUON_WGamma}. The details of the ``syst other'' and ``$e\rightarrow\gamma$'' columns are provided in Tab.~\ref{tab:systInPercentSmallSysts_ELECTRON_WGamma} and~\ref{tab:systInPercentEtogamma_ELECTRON_WGamma} respectively. The `` total'' is the total relative systematic uncertainty on $d\sigma/dP_T^{\gamma}$.}
   \begin{tabular}{|c|c|c|c|c|c|c|c|c|c|}
                   &     & \multicolumn{8}{|c|}{systematic errors}     \\
    $P_T^{\gamma}$,  & stat & \multicolumn{3}{|c|}{related to jets$\rightarrow\gamma$} &  &  &  &  & \\
    GeV           & err & $|N_{Ich}-N_{\sigma{i\eta i\eta}}|$ &$Z\gamma$ MC norm &templ. stat & SFs & lumi & $e\rightarrow\gamma$  & other & total\\ \hline
    total  & 2 & 15 & 35 & 5 & 19 & 3 & 4 & 5 & 44 \\ \hline
%    10-15 & 4 & 73 & 20 & 16 & 1 & 3 & 2 & 8 & 78 \\ \hline
    15-20 & 8 & 80 & 27 & 19 & 17 & 3 & 18 & 11 & 90 \\ \hline
    20-25 & 7 & 38 & 20 & 14 & 12 & 3 & 11 & 10 & 48 \\ \hline
    25-30 & 5 & 25 & 16 & 12 & 14 & 3 & 8 & 8 & 36 \\ \hline
    30-35 & 5 & 35 & 14 & 12 & 14 & 3 & 3 & 8 & 42 \\ \hline
    35-45 & 3 & 14 & 13 & 8 & 18 & 3 & 2 & 7 & 28 \\ \hline
    45-55 & 8 & 53 & 20 & 22 & 36 & 3 & 7 & 11 & 71 \\ \hline
    55-65 & 7 & 17 & 12 & 30 & 44 & 3 & 5 & 10 & 58 \\ \hline
    65-75 & 7 & 23 & 15 & 32 & 44 & 3 & 4 & 11 & 61 \\ \hline
    75-85 & 8 & 32 & 17 & 27 & 44 & 3 & 6 & 13 & 64 \\ \hline
    85-95 & 9 & 9 & 7 & 9 & 40 & 3 & 8 & 14 & 44 \\ \hline
    95-120 & 7 & 19 & 9 & 14 & 44 & 3 & 5 & 11 & 51 \\ \hline
    120-500 & 4 & 12 & 6 & 24 & 39 & 3 & 1 & 9 & 48 \\ \hline
  \end{tabular}
  \label{tab:systInPercent_ELECTRON_WGamma}
  \end{center}
\end{table}

\begin{table}[h]
  \scriptsize
  \begin{center}
  \caption{Relative systematic uncertainties [\%] on the $W\gamma$ differential cross section in the electron channel related to $e\rightarrow\gamma$ background estimation. The ``fit bias'' is the systematic uncertainty evaluated as the difference between results when fits are performed before and after $M_T^{W}$ selection requirement, the ```samp. stat'' is the systematic uncertainty related to limited statistical power of all MC samples involved into the background estimation, and ``total syst.'' is a quadrature sum of them. }
  \begin{tabular}{|c|c|c|c|}
    $P_T^{\gamma}$,  & total & fit  & samples\\
    GeV  & syst. & bias & stat\\ \hline
    total  & 4 & 4 & 1 \\ \hline
%    10-15 & 2 & 1 & 1 \\ \hline
    15-20 & 18 & 17 & 4 \\ \hline
    20-25 & 11 & 10 & 4 \\ \hline
    25-30 & 8 & 7 & 3 \\ \hline
    30-35 & 3 & 1 & 2 \\ \hline
    35-45 & 2 & 1 & 1 \\ \hline
    45-55 & 7 & 4 & 5 \\ \hline
    55-65 & 5 & 3 & 4 \\ \hline
    65-75 & 4 & 1 & 4 \\ \hline
    75-85 & 6 & 4 & 4 \\ \hline
    85-95 & 8 & 5 & 6 \\ \hline
    95-120 & 5 & 3 & 4 \\ \hline
    120-500 & 1 & 0 & 1 \\ \hline
  \end{tabular}
  \label{tab:systInPercentEtogamma_ELECTRON_WGamma}
  \end{center}
\end{table}

\begin{table}[h]
  \scriptsize
  \begin{center}
  \caption{Relative systematic uncertainties [\%] of smaller contributions  on the $W\gamma$ differential cross section in the muon channel (details of the column ``syst other'' from Tab.~\ref{tab:systInPercent_MUON_WGamma}). The ``syst other'' is a quadrature sum of all contributions listed in the table.}
  \begin{tabular}{|c|c|c|c|c|c|c|}
    $P_T^{\gamma}$,  & syst  & real-$\gamma$ & $A\times\epsilon$ & $M_T^W$ & PU    & unf      \\
    GeV            & other & bkg           & MC stat           & req.   & weight & MC stat \\ \hline
    total  & 4 & 1 & 0 & 1 & 4 & 1 \\ \hline
%    10-15 & 6 & 1 & 1 & 2 & 5 & 2 \\ \hline
    15-20 & 6 & 2 & 1 & 1 & 4 & 2 \\ \hline
    20-25 & 6 & 3 & 2 & 2 & 4 & 3 \\ \hline
    25-30 & 5 & 3 & 2 & 2 & 2 & 2 \\ \hline
    30-35 & 7 & 4 & 3 & 1 & 4 & 3 \\ \hline
    35-45 & 6 & 3 & 3 & 2 & 3 & 2 \\ \hline
    45-55 & 8 & 3 & 3 & 1 & 4 & 5 \\ \hline
    55-65 & 7 & 2 & 4 & 2 & 4 & 3 \\ \hline
    65-75 & 10 & 2 & 6 & 3 & 5 & 6 \\ \hline
    75-85 & 10 & 1 & 7 & 3 & 3 & 5 \\ \hline
    85-95 & 13 & 2 & 8 & 4 & 6 & 7 \\ \hline
    95-120 & 9 & 2 & 7 & 2 & 2 & 6 \\ \hline
    120-500 & 9 & 1 & 6 & 1 & 4 & 4 \\ \hline
  \end{tabular}
  \label{tab:systInPercentSmallSysts_MUON_WGamma}
  \end{center}
\end{table}

\begin{table}[h]
  \scriptsize
  \begin{center}
  \caption{Relative systematic uncertainties [\%] of smaller contributions  on the $W\gamma$ differential cross section in the electron channel (details of the column ``syst other'' from Tab.~\ref{tab:systInPercent_ELECTRON_WGamma}). The ``syst other'' is a quadrature sum of all contributions listed in the table}
  \begin{tabular}{|c|c|c|c|c|c|c|}
    $P_T^{\gamma}$,  & syst & real-$\gamma$ & $A\times\epsilon$ & $M_T^W$ & PU & unf\\
    GeV  & other & bkg & MC stat & req. & weight & MC stat\\ \hline
    total  & 5 & 2 & 0 & 1 & 4 & 2 \\ \hline
%    10-15 & 8 & 4 & 1 & 1 & 5 & 4 \\ \hline
    15-20 & 11 & 6 & 2 & 1 & 4 & 8 \\ \hline
    20-25 & 10 & 5 & 2 & 1 & 4 & 7 \\ \hline
    25-30 & 8 & 3 & 3 & 1 & 3 & 6 \\ \hline
    30-35 & 8 & 2 & 4 & 1 & 3 & 6 \\ \hline
    35-45 & 7 & 1 & 4 & 1 & 4 & 4 \\ \hline
    45-55 & 11 & 2 & 5 & 3 & 4 & 9 \\ \hline
    55-65 & 10 & 2 & 5 & 3 & 5 & 7 \\ \hline
    65-75 & 11 & 1 & 6 & 1 & 4 & 8 \\ \hline
    75-85 & 13 & 2 & 8 & 2 & 3 & 9 \\ \hline
    85-95 & 14 & 2 & 9 & 2 & 2 & 9 \\ \hline
    95-120 & 11 & 1 & 8 & 1 & 4 & 7 \\ \hline
    120-500 & 9 & 1 & 7 & 2 & 3 & 4 \\ \hline
  \end{tabular}
  \label{tab:systInPercentSmallSysts_ELECTRON_WGamma}
  \end{center}
\end{table}
