\section{Compact Muon Solenoid}
\label{sec:Exp_CMS}
\subsection{Introduction}

%CMS has a broad program with goals of direct and indirect searches of BSM physics including supersymmetric particles as well as precision measurements of various SM parameters. 

% MAY NOT NEED THIS SENTENCE HERE
%Its main feature is a large magnet to create a magnetic field of~4T to curve charged particles in the tracking system and of~2T outside to curve muons in the muon system.

CMS is a general-purpose detector designed for detecting various highly energetic particles which are produced in $pp$ collisions at the LHC~\cite{ref_CMS_TDR}. The CMS detector is cylindrically symmetric with the particle beam as the central axis. Cartesian, cylindrical and spherical coordinates are all used to describe the CMS geometry, depending on the context. The $x$-axis of the CMS points towards the center of the LHC ring while the $y$-axis points vertically up. The orientaion of the $z$-axis corresponds to the counterclockwise direction of the LHC beam (Fig.~\ref{fig:CMScoord}). Cylindrical coordinates are defined as $r=\sqrt{x^2+y^2}$, $\phi=\arctan(y/x)$. Instead of the polar angle $\theta$, it is more convenient to use the pseudorapidity $\eta=-\ln{\tan{\theta/2}}$. A pseudorapidity ranges from $\eta=-\infty$ to $\eta=+\infty$ for directions parallel to the beam axis with the value of $\eta=0$ for a direction perpendicular to the beamline. This variable is convenient for measurements because a distribution of a massless particle in $\eta$ is nearly flat. The acceptance of the CMS in $\eta$ is limited and varies from $|\eta|<2.4$ to $|\eta|<5.3$ depending on a subdetector (Fig.~\ref{fig:CMSschemView}, top).   

\begin{figure}[htb]
  \begin{center}
    {\includegraphics[width=0.65\textwidth]{../figs/Exp/CMScoord.png}}
%\includegraphics[width=0.45\textwidth]{../figs/Exp/CMScoord_eta.png}}
    \caption{CMS coordinate system. }
    \label{fig:CMScoord}
  \end{center}
\end{figure}

The detector consists, from the inner to the outer layer,  of a tracking system, an electromagnetic calorimeter (ECal), a hadronic calorimeter (HCal), a magnet and a muon system. Having the tracking system, ECal, and HCal inside of a large solenoid makes the detector compact. A slice of CMS in the $r$-$\phi$ plane is shown in Fig.~\ref{fig:CMS_slice}.

Most heavy particles produced in a collision decay immediately, and we detect its long-lived decay products including electrons, photons, muons, neutral or charged hadrons. Depending on the trace left by the particle in different subdetectors we can identify the particle type. Electrons and positrons leave curved tracks, trajectories of charged particles, in the tracking system and then induce showers in the ECal. Photons induce the same electromagnetic showers in the ECal, however, as neutral particles, they do not leave tracks in the tracking system. Hadrons normally travel through the ECal undisturbed and induce a hadronic shower in the HCal. Charged and neutral hadrons can be distinguished from each other by checking whether they leave a track in the tracking system or not. Muons are the only particles which penetrate through the ECal, the HCal and the magnet and leave tracks in the CMS muon system. Neutrinos are not directly detected by CMS.   

\begin{figure}[htb]
  \begin{center}
    {\includegraphics[width=0.98\textwidth]{../figs/Exp/CMSview1.png}\\
     \includegraphics[width=0.98\textwidth]{../figs/Exp/CMSview.png}}
    \caption{CMS detector, schematic view. Top: $r-z$ plane, bottom: $r-\phi$ plane at $z=0$~\cite{ref_CMSschemView}. }
    \label{fig:CMSschemView}
  \end{center}
\end{figure}


\begin{figure}[htb]
  \begin{center}
    {\includegraphics[width=0.98\textwidth]{../figs/Exp/CMS_Slice.png}}
    \caption{CMS detector, a schematic view of a segment in the $r-\phi$ plane at $z=0$. Traces left by muons, electrons, photons, charged and neutral hadrons in different subdetectors are shown.}
    \label{fig:CMS_slice}
  \end{center}
\end{figure}

All subdetectors are essential for the $W\gamma$ measurement, and the remainder of this chapter describes the subdetectors in greater details. Muons and electrons, which we have as final state particles in the $W\gamma$ measurement, are both affected by the CMS magnetic field, allowing the tracking system and the muon system to measure their trajectory parameters and momenta. In this dissertation we use the information of the primary vertex, the collision point, determined by the tracking system, to select our events. The tracking system also provides us information about electron and muon trajectories and momenta and distinguishes between electrons and photons. The ECal is necessary to identify electrons and photons and to measure all kinematic parameters of photons. The HCal is also used for electron and photon identification: the energy deposit in the HCal left by an electron or a photon must be very small compared to the energy deposit left in the ECal. The muon system is essential for muon reconstruction and identification.

\subsection{Magnet}
% CONTENT-WISE IS OK

A magnetic field in a particle detector is necessary to measure momenta of charged particles by track curvatures. The higher the momentum is, the less a particle trajectory is affected by the magnetic field. In CMS, the tracking system measures  momenta of all charged particles. Also, the muon system measures momenta of muons. 

The CMS magnet is placed between the HCal and the muon system. The magnet is made of superconducting wires. An electric current flowing in the wires creates a uniform field of $B=4$T inside the solenoid, for the tracking system, and also provides a smaller magnetic field of a certain configuration outside the solenoid, for the muon system. The stronger field in the tracking system is necessary because of higher track density and smaller size relative to the muon system.

\subsection{Tracking System}
%"The tracker is designed in such a way that a single track hits multiple sensors."
%    "Then the trajectory is reconstructed based on how much charge is collected on each sensor."
%       What does it mean, exactly, each of these sentences? It is not clear to me.

The tracking system measures track geometry including particle trajectories, locations of primary and secondary vertices, and momenta of charged particles. It is designed to disturb particles as little as possible so that they pass through. Therefore, just a few measurements must be enough to reconstruct the track. The accuracy of a measurement of each hit, position measurement, is~10~$\mu$m.

The tracking system consists of silicon pixels and silicon strips (Fig.~\ref{fig:tracker_slice}). Tracks that originate from proton collisions, collision tracks, start at the center and then cross the layers of the tracking system. Tracks are straight in the $r-z$ plane and curved by the magnetic field in the $r-\phi$ plane. The acceptance of the tracker system in the $r-z$ plane is geometrically limited by the absolute value of the pseudorapidity $|\eta| \leq 2.5$.

The pixel tracker is the closest subsystem of CMS to the collision point. Thus it experiences the largest particle flux: at~8~cm from the collision point the flux is about~10~million/(cm$^2$s), and the pixel detector with its~65~million sensors is capable of reconstructing all these tracks. It consists of three cylindrical layers of pixel sensors in the barrel with radii of~4~cm,~7~cm and~11~cm which are referred as barrel pixel subdetectors (BPIX) and four disks in the endcap, two disks at each side, which are referred as forward pixel subdetector (FPIX). 

%The tracker is designed in such a way that a single track hits multiple sensors. Then the trajectory is reconstructed based on how much charge is collected on each sensor. This allows us to reach a spacial resolution of~15-20~$\mu$m which is much smaller than a distance between sensors.

The strip tracker is placed right outside the pixel tracker and occupies the detector volume up to~130~cm around the beam axis. The strip tracker consists of four parts: the tracker inner barrel (TIB), the tracker inner disks (TID), the tracker outer barrel (TOB) and the tracker endcap (TEC) as shown in Fig.~\ref{fig:tracker_slice}. In the strip tracker, there are over~15,000 sensitive modules with a total number of~10~million strips. Each sensitive module consists of a set of sensors, its support structure, and readout elements.

%electric charge and amplification

%limitations

\begin{figure}[htb]
  \begin{center}
    {\includegraphics[width=0.8\textwidth]{../figs/Exp/tracker_slice.png}}
    \caption{Slice of the CMS tracking system in the $r-z$ plane.}
    \label{fig:tracker_slice}
  \end{center}
\end{figure}

\subsection{Electromagnetic Calorimeter}

The ECal is placed between the tracking system and the HCal. It is made of high-density lead tungstate crystals arranged in a barrel section and two endcap sections. The crystals are scintillators. When electrons and photons pass through the scintillators, they produce light proportional to the particle's energy. The scintillated light is then amplified by photomultipliers. After that, signals are digitized and taken away by fiber optic cables.

The ECal measures the energy of electrons and photons and parameters of their trajectories. In order to distinguish between electrons and photons, it is necessary to perform matching to the track in the tracking system. If there is a track, then the particle is an electron (or positron), otherwise the particle is a photon.

It is important for the ECal to be able to distinguish between high energy photons and pairs of lower energy photons e.g. from a $\pi^0$ decay. It is especially difficult in the endcap sections where the angle between two photon trajectories is small. For this reason, ECal preshower detectors (PS) which have~$\sim$15~times smaller granularity are located in front of the endcaps. The preshowers provide extra spatial precision. 

%Their strips are~2~mm wide compared to~3~cm wide crystals in the main volume of the ECal.

%(Why muons and hadrons don't release their energy here?)
%limitations

\subsection{Hadron Calorimeter}
 % MY ORIGINAL TEXT
%The HCal is placed right after the ECal and is the last subdetector within the magnet. The HCal measures energies of charged and neutral hadrons. In addition, the HCal determines the trajectory parameters. Match to the tracking system has to be done: if a matching track found, then it is a chagred hadron otherwise it is a neutral hadron. 

%The HCal consists of alternate layers of absorbers and scintillators. Hadrons hit brass or steel plate of absorber producing secondary particles. When emerge into the scintillator, the particles induce hadronic and electromagnetic showers. All hadrons must be stopped inside the layers of the HCal.

% CMS website

%(1) QUOTE:
%The Hadron Calorimeter (HCAL) measures the energy of “hadrons”, particles made of quarks and gluons (for example protons, neutrons, pions and kaons). Additionally it provides indirect measurement of the presence of non-interacting, uncharged particles such as neutrinos. Measuring these particles is important as they can tell us if new particles such as the Higgs boson or supersymmetric particles (much heavier versions of the standard particles we know) have been formed.
%(1) MY WORDS:
The HCal measures the energy of charged and neutral hadrons. Also, it plays a key role in the indirect detection of invisible particles like neutrinos. 

%(2) QUOTE:
%As these particles decay they may produce new particles that do not leave record of their presence in any part of the CMS detector. To spot these the HCAL must be “hermetic”, that is make sure it captures, to the extent possible, every particle emerging from the collisions. This way if we see particles shoot out one side of the detector, but not the other, with an imbalance in the momentum and energy (measured in the sideways “transverse” direction relative to the beam line), we can deduce that we’re producing “invisible” particles.
%(2) MY WORDS:
Invisible particles leave no record in CMS detector. They are detected by missing transverse energy ($E_T^{miss}$) that is determined as 
\begin{equation}\label{eq:MET}
  E_T^{miss} = - | \sum \mathbf{P_T} |,
\end{equation}
\noindent{where the summation covers all visible particles in the event. Therefore, for precise measurement of $E_T^{miss}$ it is important to capture the full energy release of all visible particles. For this purpose, HCal is designed to stop all hadrons passing through.}

The HCal consists of the barrel, endcap and forward parts: HB, HE and HF in Fig.~\ref{fig:CMSschemView}, top, respectively. Its acceptance extends to $|\eta|=3.0$ for endcaps and to $|\eta|=5.3$ for forward HCal.

%(3) QUOTE:
%To ensure that we’re seeing something new, rather than just letting familiar particles escape undetected, layers of the HCAL were built in a staggered fashion so that there are no gaps in direct lines that a familiar particle might escape through. The HCAL is a sampling calorimeter [see explanation below] meaning it finds a particle’s position, energy and arrival time using alternating layers of “absorber” and fluorescent “scintillator” materials that produce a rapid light pulse when the particle passes through. Special optic fibres collect up this light and feed it into readout boxes where photodetectors amplify the signal.   When the amount of light in a given region is summed up over many layers of tiles in depth, called a “tower”, this total amount of light is a measure of a particle’s energy.
%(3) MY WORDS:
The HCal is a sampling calorimeter. It consists of alternating layers of absorbers and scintillators. When a hadron hits an absorber, it induces a hadronic shower. The light produced by the shower is collected on optic fibers and passed to the readout system. The total amount of light released in a certain region of the HCal is a measure of hadron's energy. In addition to the energy, the HCal also reconstructs the trajectory of the hadron.   

%(4) QUOTE:
%As the HCAL is massive and thick, fitting it into “compact” CMS was a challenge, as the cascades of particles produced when a hadron hits the dense absorber material (known as showers) are large, and the minimum amount of material needed to contain and measure them is about one metre.   
%(4) MY WORDS:

%(5) QUOTE:
%To accomplish this feat, the HCAL is organised into barrel (HB and HO), endcap (HE) and forward (HF) sections. There are 36 barrel “wedges”, each weighing 26 tonnes. These form the last layer of detector inside the magnet coil whilst a few additional layers, the outer barrel (HO), sit outside the coil, ensuring no energy leaks out the back of the HB undetected.  Similarly, 36 endcap wedges measure particle energies as they emerge through the ends of the solenoid magnet.
%(5) MY WORDS:
% (Higher) The HCal consists of barrel, endcap and forward parts.  

%Lastly, the two hadronic forward calorimeters (HF) are positioned at either end of CMS, to pick up the myriad particles coming out of the collision region at shallow angles relative to the beam line. These receive the bulk of the particle energy contained in the collision so must be very resistant to radiation and use different materials to the other parts of the HCAL. 


\subsection{Muon System}

%(1) QUOTE
%Because muons can penetrate several metres of iron without interacting, unlike most particles they are not stopped by any of CMS's calorimeters. Therefore, chambers to detect muons are placed at the very edge of the experiment where they are the only particles likely to register a signal.
%(1) MY WORDS
Muons, unlike other visible particles, are not stopped by CMS calorimeters. Muons are the only particles that are registered in the muon system which is placed outside the magnet and which is the largest part of the CMS detector.

There are four concentric layers of muon detectors (stations) and iron return yoke between them. Muons induce several hits in the muon stations which are later fitted and matched to the tracking system measurements to provide the best possible resolution in the measurements of the muon's trajectory and momentum.

There are three types of muon chambers used in the CMS muon system: drift tubes (DTs), cathode strip chambers (CSCs) and resistive plate chambers (RPCs) (Fig.~\ref{fig:muonSystem}). Overall, there are~1400~muon chambers including~250~DTs,~540~CSCs and~610~RPCs.

%DTs
%(4) QUOTE:
%The drift tube (DT) system measures muon positions in the barrel part of the detector. Each 4-cm-wide tube contains a stretched wire within a gas volume. When a muon or any charged particle passes through the volume it knocks electrons off the atoms of the gas. These follow the electric field ending up at the positively-charged wire.
%By registering where along the wire electrons hit (in the diagram, the wires are going into the page) as well as by calculating the muon's original distance away from the wire (shown here as horizontal distance and calculated by multiplying the speed of an electron in the tube by the time taken) DTs give two coordinates for the muon’s position.
%Each DT chamber, on average 2m x 2.5m in size, consists of 12 aluminium layers, arranged in three groups of four, each up with up to 60 tubes: the middle group measures the coordinate along the direction parallel to the beam and the two outside groups measure the perpendicular coordinate.
%(4) MY WORDS:
The system of DTs measures positions of muons in the barrel. Each DT chamber is about~2~m by~2.5~m in size. A chamber consists of~12~layers of aluminum which are arranged in groups of four. There are up to~60~DTs in a layer. The middle group of layers measures $z$-coordinate and two other groups determine the perpendicular coordinate. The DT's volume is filled with a gas, and there is a wire inside. The DT's width is~4~cm. When a charged particle passes through the volume, it ionizes atoms, and the wire receives an electric charge. The position along the wire is registered, and the distance of the muon away from the wire is calculated providing measurements of two coordinates of the position of the muon.

%CSCs
%(5) QUOTE:
%Cathode strip chambers (CSC) are used in the endcap disks where the magnetic field is uneven and particle rates are high.
%CSCs consist of arrays of positively-charged “anode” wires crossed with negatively-charged copper “cathode” strips within a gas volume. When muons pass through, they knock electrons off the gas atoms, which flock to the anode wires creating an avalanche of electrons. Positive ions move away from the wire and towards the copper cathode, also inducing a charge pulse in the strips, at right angles to the wire direction.
%Because the strips and the wires are perpendicular, we get two position coordinates for each passing particle.
%In addition to providing precise space and time information, the closely spaced wires make the CSCs fast detectors suitable for triggering. Each CSC module contains six layers making it able to accurately identify muons and match their tracks to those in the tracker.
%(5) MY WORDS

CSCs are placed in the endcap regions. CSCs are arrays of anode wires crossed by copper cathode strips placed in a gas volume. When a charged particle penetrates the gas volume, it ionizes the gas. Electrons drift to the wires while ions move to the strips, and charge pulses are induced on wires as well as on strips. Strips are perpendicular to wires. Thus we measure two coordinates for each particle.  

% RPCs
%(6) QUOTE:
%Resistive plate chambers (RPC) are fast gaseous detectors that provide a muon trigger system parallel with those of the DTs and CSCs. RPCs consist of two parallel plates, a positively-charged anode and a negatively-charged cathode, both made of a very high resistivity plastic material and separated by a gas volume.
%When a muon passes through the chamber, electrons are knocked out of gas atoms. These electrons in turn hit other atoms causing an avalanche of electrons. The electrodes are transparent to the signal (the electrons), which are instead picked up by external metallic strips after a small but precise time delay. The pattern of hit strips gives a quick measure of the muon momentum, which is then used by the trigger to make immediate decisions about whether the data are worth keeping. RPCs combine a good spatial resolution with a time resolution of just one nanosecond (one billionth of a second).
%(6) MY WORDS: 
RPCs are parallel capacitors made of high-resistivity plastic plates with a space between them filled with gas. RPCs provide quick measurements of muon momenta and are used for triggering. A muon passing through the RPC ionizes gas atoms. Released electrons ionize more atoms inducing an avalanche. Electrodes receive signal and pass it to external strips that provide a quick measure of the muon's momentum. 


\begin{figure}[htb]
  \begin{center}
    \includegraphics[height=2.5 cm]{../figs/Exp/muonSystem_driftTubes.png}\quad\includegraphics[height=2.5 cm]{../figs/Exp/muonSystem_CSC.png}\quad\includegraphics[height=2.5 cm]{../figs/Exp/muonSystem_RPC.png}
    \caption{Components of the CMS muon system. Left to right: drift tubes (DTs), cathode strip chambers (CSCs), resistive plate chambers (RPCs).}
    \label{fig:muonSystem}
  \end{center}
\end{figure}


\subsection{Triggering and Data Aquisition}

At peak luminosity, CMS experiences forty million proton-proton collisions per second that come in bunches separated by~25~ns. It is not technically feasible to readout all these events. Moreover, we do not need most of these events for a physics measurement because most of them have not resulted from interesting physics process. We have resources to store about one hundred events out of forty million, and that is why we need a trigger system that quickly decides what the best one hundred events are.

%New events come before the events from the previous bunch crossing left the detector. To process the information from many different collisions at the same time, data is stored in pipelines.

If the triggers were too loose, and we would select one hundred events too quickly, e.g., in~1/10~s, then CMS would not be able to process the remaining~90\% of events provided by LHC in a given second and we would lose~90\% of potentially interesting events.

If the triggers were too strict, we would select, e.g.,~ten events per second, not one hundred and lose CMS's potential to store and process data by~90\% which would significantly reduce our chances for discovery and increase statistical uncertainties for precision measurements.

Thus, the challenge of the trigger system is to select the best one hundred events per second and do so quickly to be able to process every single event. To achieve this goal, a two-level trigger system was developed consisting of the Level~1 trigger (L1T) and the High Level Trigger (HLT) as shown in Fig.~\ref{fig:trigger_2level}.

L1T is a hardware based trigger (Fig.~\ref{fig:trigger_L1}). It uses information from the ECal, HCal and muon system. L1T reduces the frequency of coming events from~40~MHz to~100~kHz. Events that did not pass the L1T are lost forever while events that pass the L1T are temporarily stored to get checked by the HLT.

HLT is a software-based trigger. It uses information from all subdetectors and runs quick reconstruction and identification algorithms to determine types of particles and their kinematics. It reduces the number of events to~100~Hz. Events that did not pass HLT are lost forever. Events that pass HLT are arranged into appropriate datasets depending on HLT selection criteria they passed and stored for physics measurements.

\begin{figure}[htb]
  \begin{center}
    \includegraphics[width=0.8\textwidth]{../figs/Exp/trigger_2level.png}
    \caption{Two-level CMS trigger system.}
    \label{fig:trigger_2level}
  \end{center}
\end{figure}

\begin{figure}[htb]
  \begin{center}
    \includegraphics[width=0.8\textwidth]{../figs/Exp/trigger_L1.png}
    \caption{Level~1 CMS trigger system.}
    \label{fig:trigger_L1}
  \end{center}
\end{figure}

\subsection{Particle Flow Algorithm of Event Reconstruction}

A particle flow algorithm is used by CMS to identify and reconstruct stable particles~\cite{ref_ParticleFlowAlg}. It processes the information from all CMS subdetectors and identifies and reconstructs each stable particle in an event individually. The list of particles include muons, electrons, photons, charged and neutral hadrons. Each type of particles leaves its specific trace in the CMS detector as shown in Fig.~\ref{fig:CMS_slice}. After reconstruction of individual stable particles, jets are built, missing transverse energy $E_T^{miss}$ is determined, certain short-lived particles are reconstructed based on the list of individual stable particles in the event.

One particle can induce several different particle-flow elements in different subdetectors. The linking algorithm links these elements together producing blocks of elements. Usually, a block has between one and three elements. Links can be connections between the tracking system and PS, ECal or HCal, between PS and ECal, between ECal and HCal, and between a tracking system and a muon system. 

In each block, muons are considered first. A link between charged tracks in the tracking and muon systems comprise a global muon which produces one ``particle-flow muon''. The corresponding track in the tracking system is removed from the block and corresponding energy deposits are subtracted from ECal and HCal. Then electrons are reconstructed and identified using the tracking system and ECal. The corresponding tracks and ECal clusters are removed from the block. Remaining tracks and clusters are considered more carefully to identify charged hadrons, neutral hadrons, and photons.

When all particles in the event are reconstructed and identified, $E_T^{miss}$ is determined. $E_T^{miss}$ is used in physics measurements as a measure of $P_T$ of neutrinos and other invisible particles in the event. Fake $E_T^{miss}$ can originate from particles that did not fall into the detector acceptance, particles that they did not reach the tracking system because their momenta was too low and, therefore, track curvature was too high, momenta mismeasurement, particle misidentification, cosmic rays particles, and machine background.

In the measurement of this dissertation particle flow muons, electrons, photons, and $E_T^{miss}$ are used for all the major steps of the cross section measurement including event selection, background subtraction, various corrections, and determination of phase space restrictions and bin boundaries. Each step is described in greater details in~Ch.~\ref{sec:AN_WgMeas}. 

%Acceptance: particles which are too collinear and go to pipe; particles which get curved too strongly
