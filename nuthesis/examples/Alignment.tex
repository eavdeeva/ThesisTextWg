\chapter{CMS Tracker Alignment} % 5-10 pages
\label{sec:alignment}

In the presence of a constant magnetic field, a charged particle has a helical trajectory which can be parametrized by five constants in three dimensions. While a charged particle travels through a tracking system, the tracking system detects hits. A reconstruction algorithm determines the track parameters by fitting the positions of hits assuming a helical trajectory. That allows the reconstruction of the full geometry of the track as well as the corresponding particle momentum, and to determine whether the particle came from the point of the $pp$ collision or decay of a secondary particle.

High precision track reconstruction is necessary for accurate measurements of particle kinematics. Better location uncertainty leads to higher precision of a measurement of the track parameters. The location uncertainty depends on our knowledge of the positions and orientations in the space of the tracking system modules. For example, the hit resolution in the CMS pixel detector is~$\sim$10$~\mu$m in the $r$-$\phi$ plane and~$\sim$30$~\mu$m in the $r$-$z$ plane~\cite{ref_trackerPerformance}. 

When the modules of the pixel detector are mounted, their positions are known with precision of~$\sim$200$~\mu$m. To take full advantage of the resolution of 10$~\mu$m, we need to know positions of modules at the accuracy of the single hit resolution. The procedure for the determination of the module locations and orientations is called the tracker alignment. The approach used for the tracking alignment in CMS is described in Ch.~\ref{sec:alignmentAlg}.

The procedure of tracker alignment is essential for the momentum measurement of all charged particles including electrons and muons that are the final state particles of the measurement of this dissertation as well as for the determination of the position of the primary vertex, the interaction point of a $pp$ collision that caused a given process. The measurement of this dissertation is based on data collected in~2012 while the author of this dissertation participated in the alignment of the tracking system in~2015 (Ch.~\ref{sec:alignmentResults}). The results of~2015 alignment are not used for the measurement of this dissertation but are used for all CMS physics measurements of~2015 data including $W\gamma$ measurement at $\sqrt{s}=$13~TeV. 

