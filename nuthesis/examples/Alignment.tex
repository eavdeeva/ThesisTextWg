\chapter{CMS Tracker Alignment} % 5-10 pages
\label{sec:alignment}

In a presence of a constant magnetic field, a charged particle has a helical trajectory which can be parametrized by five constants in three dimensions. While a charged particle travel through a tracking system, the tracking system detects hits. A reconstruction algorithm determines the track parameters by fitting the positions of hits assuming the helical trajectory. That allows to reconstruct the full geometry of track as well as to determine a particle momentum as well as to determine whether the particle came from the point of primary $pp$ collision or from decay of a secondary particle.

High precision track reconstruction is essencial for particle identification and accurate measurements of particle kinematics. Better hit resolution and location uncertainty lead to higher precision of a measurement of the track parameters. The location uncertainty depends on our knowledge of the positions and orientations in space of the tracking system modules. For example, the hit resolution in the CMS pixel detector is~$\sim$15~$\mu$m. When the modules of the pixel detector are mounted, their positions are known with precision of~$\sim$200~$\mu$m. Thus, we need to know positions of modules~20~times better than they are known when mounted. The procedure of the determination of the modules locations and orientations is called the tracker alignment. The approach used for the tracking alignment in CMS is described in Ch.~\ref{sec:alignmentAlg}.

The procedure of tracker alignment is essential for the momenta measurement of all charged particles including electrons and muons that are the final state particles of the measurement of this dissertation as well as the determination of the position of the primary vertex. The author of this dissertation participated in the alignment of the tracking system in~2015 (Ch.~\ref{sec:alignmentResults}), thus, the results are not used for the measurement of this dissertation which is based on~2012 data but are used for all CMS physics measurements of~2015 data including $W\gamma$ measurement at $\sqrt{s}=13$~TeV. 

