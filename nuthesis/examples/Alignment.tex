\chapter{CMS Tracker Alignment} % 5-10 pages
\label{sec:alignment}
A tracking system detects hits produced by a charged particle traveling through the detector that allows to reconstruct the full geometry of track as well as to determine a particle momentum. In a presence of a constant magnetic field the particle has a helical trajectory. A reconstruction algorithm determines the track parameters by fitting the positions of hits assuming the helical trajectory which can be defined by five parameters.

High precision track reconstruction is essencial for particle identification and accurate measurements of particle kinematics. Smaller hit resolution and location uncertainty lead to higher precision of a measurement of the track parameters. The location uncertainty depends on our knowledge of the positions and orientations in space of the tracking system modules. The hit resolution in the CMS pixel detector is~$\sim$15~$\mu$m. When the modules are mounted, their positions are known with precision of~$\sim$200~$\mu$m. Thus, we need to know positions of modules~20~times better than they are known when mounted. The procedure of the determination of the modules locations and orientations is called the tracker alignment.

