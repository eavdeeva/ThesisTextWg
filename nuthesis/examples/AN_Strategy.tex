\section{Measurement Strategy}
\label{sec:AN_WgMeasStrategy}

The process of the cross section measurement is a sequence of steps summarized in Tab.~\ref{tab:analysisOutline}. First, we select events and obtain the number of selected events (Ch.~\ref{sec:AN_Selection}). In Tab.~\ref{tab:analysisOutline} the total number of selected events is $N_{sel}$ and numbers of selected events in $P_T^{\gamma}$ bins are $N_{sel}^j$, where $j$ is a bin number. The selected sample contains signal as well as background events. The next step is to subtract the background. This step results in background subtracted yields of signal events $N_{sign}$ and $N_{sign}^j$ (Ch.~\ref{sec:BackgroundSubtraction}). After that, we apply an unfolding procedure that corrects for detector resolution effects in measurement of photon transverse momentum ``detector resolution unfolding'', (Ch.~\ref{sec:Unfolding}) and obtain yields within acceptance and selection restrictions: $N_{A\times\epsilon}^i = U_{ij} \cdot N_{sign}^j$, where $U_{ij}$ is the unfolding operator. The detector resolution unfolding is only relevant for the measurement of the differential cross section, not the total cross section. Then corrections for kinematic and geometrical acceptance and reconstruction and selection efficiency are applied (Ch.~\ref{sec:AccXEff}). Finally, we divide the measured number of events by integrated luminosity recorded by CMS and, in the case of the differential cross section, by the width of $P_T^{\gamma}$ bins. This results in the total and differential cross section (Ch.~\ref{sec:AN_CrossSection}). Each step has its systematic uncertainties associated with it, and we estimate their contributions to the final results (Ch.~\ref{sec:Systematics}).

\begin{table}[h]
  \small
  \begin{center}
  \caption{Measurement steps and algebraic representations of the steps for the differential (``$d\sigma/dP_{T}^{\gamma}$'') and total (``$\sigma$'') cross section measurements. }
  \begin{tabular}{|c|c|c|}
    \hline
    Step & $d\sigma/dP_{T}^{\gamma}$ & $\sigma$ \\ \hline
    select events & $N_{sel}^j$ &    $N_{sel}$       \\ \hline
    subtract background & $N_{sign}^j = N_{sel}^j - N_{bkg}^j$ &    $N_{sign}=N_{sel}-N_{bkg}$       \\ \hline
    unfold   & $N_{A\times\epsilon}^i = U_{ij} \cdot N_{sign}^j$ &    $-$       \\ \hline
    correct for the acceptance and efficiency & $N_{true}^i = \frac{N_{A\times\epsilon}^i}{(A \times\epsilon)^i}$ &  $N_{true}=\frac{N_{sign}}{A\times\epsilon}$       \\ \hline
    divide by luminosity and bin width & $ \left( \frac{d\sigma}{dP_{T}^\gamma} \right) ^i = \frac{N_{true}^i}{L \cdot (\Delta P_T^\gamma)^i}$  &  $\sigma = N_{true}/L$       \\ \hline
    estimate systematic uncertainties &  &         \\ \hline
  \end{tabular}
  \label{tab:analysisOutline}
  \end{center}
\end{table}

At first, we perform the measurement in a blinded way. The purpose of blinding is to avoid unintended biasing of our results in any direction. Our blinding strategy is the following:
\begin{itemize}
  \item for $p_T^{\gamma}<$45~GeV: use 100\% of data; and
  \item for $p_T^{\gamma}>$45~GeV: use 5\% of data (every 20$^{th}$ event).
\end{itemize}
\noindent{The threshold of $p_T^{\gamma}=$45~GeV is chosen because below that we do not expect any new physics, and the percentage of~5\% is chosen because this amount of data gives us such a large statistical uncertainty so that we would not notice any new physics if it were there. After the measurement procedure is fully established, we perform the measurement using our full dataset (``unblinded'' measurement). All plots in this dissertation are shown for the final, unblinded, stage of the measurement.}

% REPHRASE THE UNBLINDING STUFF

A brief description of the software tools used and developed for the measurement is available in App.~\ref{sec:Code}.
