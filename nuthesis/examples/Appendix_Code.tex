% APPENDIX
\chapter{Code and Software}
\label{sec:Code}

% ROOT, RooFit
% RooUnfold
% MCFN, MadGraph
% my code

% CMSSW
The CMS software (CMSSW)~\cite{ref_CMSSW} is the tool developed to process all CMS responses, reconstruct particles and prepare data in a condition convenient for the final physics measurements. CMSSW is mostly written in C++ and Python programming languages. It has hundreds of contributors that use GITHUB~\cite{ref_GITHUB} to share their work. All CMS physics measurements use CMSSW.

The procedure of the tracker alignment and validation described in Ch.~\ref{sec:alignment} is also a part of the CMSSW although the Millepede-II algorithm itself is implemented in the external software tool.

%ROOT, RooFit
The samples for the physics measurements are stored in a format of ROOT trees. The ROOT tree contains multiple parameters for each entry and allows easy access to all parameters. These properties make it convenient to use ROOT trees for particle physics measurements where, usually, one entry corresponds to one event or one candidate. The ROOT trees provided by reconstruction algorithms of CMSSW are referred as ``tuples''. Tuples are further processed by different large physics subgroups that prepare ``ntuples''. Ntuples store only information that is necessary for a specific class of measurements and arrange it in a more convenient way for this specific class of measurements.  

% ggNtuples
The author of this dissertation used ``ntuples'' prepared by Central Taiwan University and Kansas State University groups mostly for various diboson and triboson measurements. The code of the program that prepares the ``ntuples'' is available at~\cite{ref_ggNtuplizer}.

% my code
The code for the CMS $W\gamma$ measurement at $\sqrt{s}=8$~TeV was written by the author of this dissertation using C++ language, ROOT and RooFit~\cite{ref_RooFit} packages, $RooUnfold$~\cite{ref_RooUnfold} class for the detector resolution unfolding, and $RooCMSShapePdf$~\cite{ref_RooCMSShapePdf} for $e\rightarrow\gamma$ background estimation. Auxiliary shell scripts are used to run the chain of C++ programs corresponding to separate physics measurement steps. The code is available at~\cite{ref_GITHUB}.

% cross checks
Several cross check were performed with other collaborators to make sure the code is free of major errors. Especially the event selection and background estimation for the electron channel is fully implemented by both Kansas State University group and the author of this dissertation in separate frameworks. These procedures are carefully cross checked between two developers.  
