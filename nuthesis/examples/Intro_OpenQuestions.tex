\section{Open Questions of the Standard Model}


While the SM is an accurate description of all particle physics experimental results, there are certain phenomena which are not included into the SM. In this subsection we discuss some of them.

The gravitational interactions do not fit into the SM. It is the open question whether the quantum theory of gravity is possible and whether there is a mediator of the gravitational interactions. Also, it is not known why the gravitational force is so much weaker than the other forces. One possible explanation comes from a theory which predicts extra spatial dimensions beyond the three we experience (e.g. the string theory). In this case, it is possible that the gravitational force is shared with other dimensions and only a fraction is available in our three dimensions.

Another mystery of the universe is its composition: it is known from the studies of the gravitational effects that our universe consists of dark energy by 68\%, of dark matter by 27\% and of baryon matter only by 5\%~\cite{ref_NASA}. The dark energy resists the gravitational attraction and accelerates the expansion of the universe, and is not detectable by any effects except gravitational. The understanding of dark energy is a question of general relativity rather than particle physics. The dark matter, however, likely consists of particles and therefore is a subject of particle physics. It does not radiate and that is why it cannot be detected by telescopes. The nature of the dark matter is not known but its constituents must be very stable to remain since the Big Bang. The theory of the supersymmetry which is unifying fundamental particles and mediators predicts many of new heavy particles and the lightest supersymmetric particle, the neutralino, is a good candidate for dark matter.

One more open question is the reason for the matter/antimatter asymmetry. Matter and antimatter should have been created in the same amount at the moment of the Big Bang. Most of it has annihilated but because of asymmetry, there was more matter than antimatter which led to the state of the Universe we observe now. There is a phenomenon of the CP-violation in weak interactions observed and described which predicts the asymmetry at a certain level. However, the effect of the CP-violation is not large enough to account for the observed amount of the matter and, therefore, the total matter/antimatter asymmetry remains unexplained. 

The measurement of the $W\gamma$ production in $pp$ collisions has a goal to both test the SM and search for the BSM physics. We measure a cross section differential in the component of the photon momentum, transverse to the beamline (referred as photon transverse momentum, or $P_T^{\gamma}$). The low $P_T^{\gamma}$ region is not expected to be affected by any new physics and must agree well with the SM predictions while the high $P_T^{\gamma}$ region may indicate an existence of new physics if there is an enhancement over the SM predictions. The excess would be indirect evidence of the BSM particles like supersymmetric particles or additional gauge bosons which could be part of the explanation of the dark matter presence or difference in magnitudes of different interactions. More theoretical details about the SM description of W$\gamma$ process as well as possible BSM physics are given in Ch.~\ref{sec:WgAbout}.    

%Grand unification and super unification

