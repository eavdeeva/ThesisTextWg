\section{Data and Monte Carlo Samples}
\label{sec:DataAndMC}

The data sample we use in this analysis was recorded by the CMS experiment in~2012 in LHC $pp$ collisions at $\sqrt{s}=8$~TeV. To select $W\gamma$ events, we use data collected by single muon and single electron triggers. The single muon trigger requires that in each event there is at least one reconstructed muon with $P_T^{\mu}>24$~GeV and $|\eta|<2.1$ which also satisfies certain requirement of isolation from other particles. The single electron trigger requires at least one reconstructed electron with $P_T^{e}>27$~GeV which also passes a certain set of identification requiremets.

In addition to $W\gamma$-selected data sample, we also prepare $Z\gamma$-selected data sample which is used for the background estimation (Ch.~\ref{sec:BackgroundSubtraction}) and for cross checking purpose (App.~\ref{sec:ZgCheck}). To select $Z\gamma$ events, we use double muon and double electron triggers. The double muon trigger requires a presence of at least two reconstructed muons with $P_T^{\mu}>17$~GeV and $P_T^{\mu}>8$~GeV per event. The double electron trigger requires a presence of at least two reconstructed electrons with $P_T^{e}>17$~GeV and $P_T^{e}>8$~GeV which also satisfy several other criteria of electron's quality.

%_HLT_Mu17_Mu8_v
%_HLT_Ele17_Ele8_v, many other conditions

 % are used as the signal samples while double muon and double electron datasets are used for data-driven background estimation. 

%The data is collected by single electron ($p_T>27$~GeV, WP 80) 
%and single muon ($p_T>24$~GeV, $|\eta|<2.1$) triggers

% Only runs and luminosity sections certified by CMS are considered in the measurement, 
% which means that good functioning of all CMS sub-detectors is required.

%THE IDEAS FROM THIS PARAGRAPH NEED TO BE REPACKAGED
%All simulation samples (often referred as Monte Carlo or MC samples) used in this measurement are generated with MadGraph~\cite{ref_MadGraph} and reconstructed centrally by the CMS simulation team. Information regarding signal and background simulated samples used for our measurement is given in Tab.~\ref{tab:mc_bkg_samples} alongside with the corresponding cross sections. The $Z$+jets process is often referred as Drell-Yan + jets or DY+jets.

% The simulated samples are reweighted to represent the distribution of the number of $pp$ interactions per bunch crossing (pileup or PU), 
% as measured in the data.
% MAYBE, JUST EXPLAIN PILEUP IN SELECTION CHAPTER

\begin{table}[h]
  \small
  \begin{center}
    \caption{Summary of simulated samples used in the measurement.}
    \begin{tabular}{|l|l|l|}
      \hline
      Process                              & $\sigma$, pb         \\ \hline
      $W\gamma \rightarrow l\nu\gamma$     & 553.92     \\ \hline         % (NLO)
      $W$+jets$ \rightarrow l\nu + jets$   & 36257.2   \\  \hline % (NNLO)
      $Z$+jets$ \rightarrow ll + jets$     & 3503.71         \\ \hline
      $t\bar{t}$+jets$\rightarrow 1$l+X    & 99.44    \\ \hline % (NNLO)
      $t\bar{t}$+jets$\rightarrow 2$l+X    & 23.83         \\ \hline
      $t\bar{t}\gamma$                     & 1.444          \\ \hline
      $Z\gamma \rightarrow ll\gamma$       & 171.62           \\ \hline
    \end{tabular}
    \label{tab:mc_bkg_samples}
  \end{center}
\end{table} 

The NLO cross section of $W\gamma$ was calculated with the MCFM in the same phase space for which the $W\gamma$ sample was generated. The NNLO contribution is estimated to be~19\%-26\% of the NLO value~\cite{ref_theory_NNLO}.

The cross section of $Z\gamma$ measured by CMS is taken~\cite{ref_Zg8TeV} and expanded to the phase space of the $Z\gamma$ generated MC sample. The resulting cross section is found to be~$171.62$~pb.
