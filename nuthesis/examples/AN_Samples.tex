\section{Data and Monte Carlo Samples}
\label{sec:DataAndMC}

The data sample we use in this analysis was recorded by the CMS experiment in~2012 in LHC $pp$ collisions at $\sqrt{s}=8$~TeV. To select $W\gamma$ events, we use data collected by single muon and single electron triggers. The single muon trigger requires that in each event there is at least one reconstructed muon with $P_T^{\mu}>24$~GeV and $|\eta|<2.1$ which also satisfies certain requirement of isolation from other particles. The single electron trigger requires at least one reconstructed electron with $P_T^{e}>27$~GeV which also passes a certain set of identification requiremets, including isolation. Such trigger choice maximize our chances to select $W\gamma$ events out of dominant multijets events.

In addition to $W\gamma$-selected data sample, we also prepare $Z\gamma$-selected data sample which is used for the background estimation (Ch.~\ref{sec:BackgroundSubtraction}) and for cross checking purpose (App.~\ref{sec:ZgCheck}). To select $Z\gamma$ events, we use double muon and double electron triggers. The double muon trigger requires a presence of at least two reconstructed muons with $P_T^{\mu}>17$~GeV and $P_T^{\mu}>8$~GeV per event. The double electron trigger requires a presence of at least two reconstructed electrons with $P_T^{e}>17$~GeV and $P_T^{e}>8$~GeV which also satisfy several other criteria of electron's quality.

%_HLT_Mu17_Mu8_v
%_HLT_Ele17_Ele8_v, many other conditions

 % are used as the signal samples while double muon and double electron datasets are used for data-driven background estimation. 

%The data is collected by single electron ($p_T>27$~GeV, WP 80) 
%and single muon ($p_T>24$~GeV, $|\eta|<2.1$) triggers

% Only runs and luminosity sections certified by CMS are considered in the measurement, 
% which means that good functioning of all CMS sub-detectors is required.

%THE IDEAS FROM THIS PARAGRAPH NEED TO BE REPACKAGED
All simulated samples (often referred as Monte Carlo or MC samples) used in this measurement are generated with MadGraph~\cite{ref_MadGraph} and reconstructed centrally by the CMS simulation team. Information regarding MC samples used for our measurement is given in Tab.~\ref{tab:mc_bkg_samples} alongside with the corresponding cross sections at~$8$~TeV. All cross section are calculated with kinematic restrictions matching to the kinematic restrictions of the samples. 

When we select $W\gamma$ events, certain number of events from other processes pass the selection criteria too. Tab.~\ref{tab:mc_bkg_samples} contains all sources that significantly contribute to the selected sample. $W\gamma \rightarrow l\nu\gamma$ contains $W\gamma \rightarrow \mu\nu\gamma$ and $W\gamma \rightarrow e\nu\gamma$ which are our signal samples and $W\gamma \rightarrow \tau\nu\gamma$ which is our background sample. The other samples listed in Tab.~\ref{tab:mc_bkg_samples} are background samples. They are used for the background estimation and cross checking as explained in detail in the remainder of the chapter.

The $Z$+jets$ \rightarrow ll $+jets process is interchangeably used in this dissertation as Drell-Yan + jets$ \rightarrow ll $+jets or DY+jets$ \rightarrow ll $+jets. The requirement on the invariant mass of the final state lepton pair is $M_{ll}>50$~GeV. 

Samples $W$+jets and DY+jets are prepared in a way they do not overlap with $W\gamma$ and $Z\gamma$ samples.

\begin{table}[h]
  \small
  \begin{center}
    \caption{Summary of simulated samples used in the measurement.}
    \begin{tabular}{|l|l|l|}
      \hline
      Process                              & Type & $\sigma$, pb  \\ \hline
      $W\gamma \rightarrow l\nu\gamma$     & signal & 554   \\ \hline %(NLO)
      $W$+jets$ \rightarrow l\nu $+jets   & background & 36257  \\ \hline %(NNLO)
      $Z$+jets$ \rightarrow ll $+jets     & background & 3504  \\ \hline
      $t\bar{t}$+jets$\rightarrow 1l$+X    & background & 99    \\ \hline %(NNLO)
      $t\bar{t}$+jets$\rightarrow 2l$+X    & background & 24    \\ \hline
%      $t\bar{t}\gamma$                     & background & 1    \\ \hline
      $Z\gamma \rightarrow ll\gamma$       & background & 172   \\ \hline
    \end{tabular}
    \label{tab:mc_bkg_samples}
  \end{center}
\end{table} 

The NLO cross section of $W\gamma$ was calculated with the MCFM in the same phase space for which the $W\gamma$ sample was generated. The NNLO contribution is estimated to be~19\%-26\% of the NLO value~\cite{ref_theory_NNLO}. We use an NLO cross section value, and the NNLO estimate is used as an systematic uncertainty of the $W\gamma$ sample when it is applicable. 

The uncertainty on normalization of the $Z\gamma$ sample gives a significant contribution to the uncertainty of a the measurement because $Z\gamma$ MC sample is used to estimate the most significant background (Ch.~\ref{sec:BackgroundSubtraction_jtog}). MCFN provdes a value of the cross section with uncertainty of 20\%. To minimize the uncertainty, we use a cross section of $Z\gamma$ measured by CMS~\cite{ref_Zg8TeV} and recalculated it for the phase space of the generated $Z\gamma$ MC sample. 

The $Z\gamma$ cross section of $\sigma = 2073 \pm 95 \pm \11 \pm \53$~fb has been quoted in the phase space described in~\cite{Zg8TeV}. To determine the measured cross section in the generator phase space, the following formula was used:
\begin{equation}
\sigma_{ps1} = \sigma_{ps2}^{meas.} \cdot \frac{N_{ps1}^{MC}}{N_{ps2}^{MC}},
\end{equation}
\noindent{where $\sigma_{ps2}^{meas.}$ is the cross section measured in~\cite{Zg8TeV}, $N_{ps1}^{MC}$ and $N_{ps2}^{MC}$ are numbers of events in the full phase space of $Z\gamma$ MC samples and in the phase space corresponding to the measured cross section  $\sigma_{ps2}^{meas.}$, $\sigma_{ps1}$ is the resulting cross section of the $Z\gamma$ sample in its full phase space. }

The resulting $Z\gamma$ cross section is found to be~$172$~pb. Uncertainties on normalizations of other samples do not contribute significantly to the uncertainty of the measurement, therefore, we use MCFN values for them.

%HOW ZGamma Cross section was mapped to a different phase space (from AN):
%The cross section of Zg was computed using as inputs MCFM cross section values as well as the precise CMS measurementt~\cite{Zg8TeV} using the following procedure. The Z$\gamma$ cross section of $\sigma = 2073 \pm 95 \pm \11 \pm \53$ fb has been quoted in the phase space described in ~\cite{Zg8TeV}. To determine the measured cross section in the generator phase space, the following formula was used:\\
%$\sigma_{narrow phase space}^{meas.}/\sigma_{wide phase space}^{meas.} = \sigma_{narrow phase space}^{MCFM}/\sigma_{wide phase space}^{MCFM}$.\\
%The $\sigma_{narrow phase space}^{MCFM}$ was determined by computing how many events are falling into the narrow phase space and scaling it to $\sigma_{wide phase space}^{MCFM}$=159.12 pb. The $\sigma_{narrow phase space}^{MCFM}$ was found to be 1933 fb and 1911 fb for the muon and electron channels respectively and the average of 1922 fb was used in the formula. The $\sigma_{wide phase space}^{meas.}$ was found to be 171.62 pb.

At the instantaneous luminosities of LHC in~2012, as a rule, multiple $pp$ interactions occured per bunch crossings. Multiple interactions are also simulated in the MC samples. However, MC samples are usually produced before data collection is finished, and in the end have to be rewithed so that the distribution of the number of interactions in simulated sample matches the data.

The pileup (PU) weights are assigned on each event in each MC sample. Figure~\ref{fig:DATAvsMC_nVtx} shows the distribution of the number of vertices of the $Z\gamma$-selected dataset overlaid with $Z\gamma$ and DY+jets MC samples in the muon channel before (left) and after (right) the PU reweighting of the MC samples. The $Z\gamma$-selected dataset is composed mainly of two sources and the $Z\gamma$ cross section is already measured and published by CMS~\cite{ref_Zg8TeV}, thus, we understand the distributions of $Z\gamma$-selected samples better the distributions of the $W\gamma$-selected samples. Thus, we use $Z\gamma$-selected dataset to validate the procedure of PU reweighting which is the same for $Z\gamma$-selected and $W\gamma$-selected MC samples.

\begin{figure}[htb]
  \begin{center}
   \includegraphics[width=0.45\textwidth]{../figs/figs_v11/MUON_ZGamma/PrepareYields/c_TotalDATAvsMC_EtaCommon__nVtx_noPU.png}\includegraphics[width=0.45\textwidth]{../figs/figs_v11/MUON_ZGamma/PrepareYields/c_TotalDATAvsMC_EtaCommon__nVtx.png}
  \caption{Number of vertices of $Z\gamma$ candidates in the muon channel. Data vs MC. Left: no PU reweighting applied, right: PU reweighting applied. }
  \label{fig:DATAvsMC_nVtx}
  \end{center}
\end{figure}
