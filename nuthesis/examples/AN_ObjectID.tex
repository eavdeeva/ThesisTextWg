\section{Event and Object Selection}
\label{sec:AN_Selection}

%In this chapter we document the electron, muon, and photon identification and isolation criteria, $E_T^{MET}$ criteria, and provide the results of comparing simulation with data.
\subsection{Object Selection}
\label{sec:AN_ObjectSelection}

We select events with a muon and a photon in the final state for the muon channel and events with an electron and a photon in the final state for the electron channel. 

We apply selection requirements on transverse momenta of $P_T^{\mu}>25$ GeV on muons,  $P_T^e>30$~GeV on electrons and $P_T^{\gamma}>15$~GeV on photons. In addition, electrons and photons must be within barrel (EB) or endcap (EE) sections of the Ecal which correspondto pseudorapidity ranges of $|\eta^{e,\gamma}| < 1.4442$ and $1.566 < |\eta^{e,\gamma}| < 2.5$, respectively. Muons must be within $|\eta^{\mu}|<2.1$. Selection requirements on $P_T^{\mu}$, $\eta^{\mu}$, and $P_T^e$ are determined by the trigger requirements, $\eta^{e,\gamma}$ criteria are determined by the geometrical limitations of the detector acceptance, and $P_T^{\gamma}>15$~GeV is the phase space requirement.

CMS Particle Object Group (POG) provides their recommendations for object identification~(ID) criteria for any given period of data collection. Recommendations for~2012 data include two sets of muon~ID criteria: "Tight" and "Loose" and four sets of electron and photon~ID criteria: "Tight", "Medium", "Loose" and "Veto".

For muon selection, we apply "Tight"~ID criteria. We consider electrons passing the "Tight" ID criteria and photons passing the modified "Medium" ID criteria. The modification of the photons ID criteria was studied in the $W\gamma\gamma \rightarrow l\nu\gamma\gamma$ measurement~\cite{ref_Wgg8TeV}.  %No restrictions on the maximum number of the final state photons are applied, however, 

To reduce backgrounds from the process with two or more leptons, such as $Z\gamma\rightarrow l l \gamma$ process, in the muon channel, we reject all events that have the second reconstructed muon candidate with $P_T^{\mu}>10$ GeV and $|\eta|^{\mu}<2.4$, and in the electron channel, we reject events which have the second reconstructed electron candidate with $p_T^e>10$ GeV and satisfying the "Veto"~ID criteria.

Selection criteria are applied consistently on the data sample as well as on all MC samples. The selection efficiency may differ between data and MC. The ratios data and MC efficiencies are called the scale factors. The scale factors for the selection criteria recommended are provided by CMS POG. For the modified photon~ID criteria, the appropriate changes to the POG-recommended scale factors were applied derived by the $W\gamma\gamma$ team~\cite{ref_Wgg8TeV}.

\subsection{Event Level Selection}
\label{sec:AN_Selection_EventLevel}

In the final state of the $W\gamma\rightarrow l\nu\gamma$ process, there is a lepton, a photon, and a neutrino. Because of that, we select events with exactly one lepton (muon or electron), a photon, both originating from the primary vertex, and with the significant missing transverse energy $E_T^{miss}$. The selection criteria for the individual electrons, muons and photons are described in Ch.~\ref{sec:AN_ObjectSelection}.

The standard tool to detec a particle that decays is to reconstruct its invariant mass out of its decay products. Decay products of a $W$ boson are a charged lepton and a neutrino. CMS does not detect neutrino, it only measures the missing momentum in the plane, transverse to the beamline, which can be partially associated with a neutrino. The transverse momentum is described by two parameters: $E_T^{miss}$ and $\phi^{miss}$. Because we do not have an estimate of the longitudial component of a neutrino, we cannot construct an invariant mass of a $W$ boson. Instead, we construct its transverse mass:
\begin{equation}
M_T^W=\sqrt{(2  P_T^{l}  E_T^{miss}  (1-\cos{(\phi^{l}-\phi^{miss})}))},
\end{equation}
\noindent{where $P_T^l$ is a lepton transverse momentum, $\phi^{l}$ is an azimuthal angle of the lepton momentum, and $\phi^{miss}$ is an azimuthal angle of the missing transverse momentum. To enchance contribution from $W\gamma$ compared to background processes without a final state neutrino, we require $M_T^W>40$~GeV. Value of $40$~GeV was recommended by the CMS Standard Model Physics (SMP) group because the same requirement was used in $W\gamma\gamma$ measurement. The $M_T^W$ distribution is shown in Fig.~\ref{fig:DATAvsMC_WMt}. Photons with $P_T^{\gamma}<45$~GeV are selected for this plot because for such photons we do not expect them to be result of a new physics process.}

After the listed selection criteria are applied, significant background from DY+jets in the electron channel remains. This background is caused by one of the electrons misidentified as a photon. Its contribution is the most significant around the invariant mass of the electron-photon system $M_{e\gamma}$ close to the mass of the $Z$ boson (Fig.~\ref{fig:DATAvsMC_Mpholep1})because the distribution of $M_{ee}$ in the $Z\rightarrow e e$ decay is peaking at the value of the Z boson mass. To reduce this background, we apply $Z$-mass window selection criterion, more specifically, events with $70$~GeV$<M_{e\gamma}<110$~GeV are rejected. 

In addition to $W\gamma$-selected datasets, we also prepare $Z\gamma$-selected datasets in muon and electron channels. Selection requirements include at least two muons (or electrons) and at least one photon in the final state. Kinematis and identification requirements on the objects are the same as for the $W\gamma$ selection. Unlike in $W\gamma$, in $Z\gamma$ selection in the electron channel, photons are required to pass ``Medium'' ID without any modifications. Invariant mass of the final state lepton pair is required to be $M_{ll}>50$~GeV. Finally, a separation between photon and each lepton must be $\Delta R>0.7$.

Finally, the separation $\Delta R=\sqrt{({\Delta\phi}^2+{\Delta\eta}^2)}$ between the final state lepton and photon is required to be $\Delta R(l,\gamma)>0.7$ to enhance the TGC contribution. In case if there is more than one photon in the selected event, the candidate with the photon of the highest~$P_T^{\gamma}$ is selected. 

In addition to $W\gamma$-selected datasets, we also prepare $Z\gamma$-selected datasets in muon and electron channels. Selection requirements include at least two muons (or electrons) and at least one photon in the final state. Kinematis and identification requirements on the objects are the same as for the $W\gamma$ selection. Unlike in $W\gamma$, in $Z\gamma$ selection in the electron channel, photons are required to pass ``Medium'' ID without any modifications. Invariant mass of the final state lepton pair is required to be $M_{ll}>50$~GeV. Finally, a separation between photon and each lepton is required to be the same as in the $W\gamma$ selection: $\Delta R>0.7$. 

\subsection{Selected Events}

%Distributions of $P_T^{\gamma}$ and $M_T^W$ of the selected events are shown in Fig. \ref{fig:DATAvsMC} and \ref{fig:DATAvsMC_WMt}. 
%The is a large discrepancies in all the distributions and therefore the data-driven background estimates are necessary.\\

After the selection procedure, 175889 and 85643 events survived in the muon and electron channels, respectively. These events are used for the total and differential cross setion measurements with respect to $P_T^{\gamma}$. Distributions of $P_T^{\gamma}$ of the selected events are shown in Fig.~\ref{fig:DATAvsMC} and documented in Tab.~\ref{tab:yields_Wg_to_munu__Barrel_}-\ref{tab:yields_Wg_to_enu__Endcap_}. The plots and tables include information about the underflow $P_T^{\gamma}$ bin ($10-15$~GeV). The measurement in this bin is used for the detector resolution unfolding (Ch.~\ref{sec:Unfolding}). 

There are large discrepancies in all the distributions as shown in Fig.~\ref{fig:DATAvsMC_WMt}-\ref{fig:DATAvsMC}. Therefore, the data-driven background estimates are necessary.

\begin{figure}[htb]
  \begin{center}
   \includegraphics[width=0.5\textwidth]{../figs/figs_v11/MUON_WGamma/PrepareYields/c_TotalDATAvsMC_EtaCommon__WMtVERY_PRELIMINARY.pdf}\includegraphics[width=0.5\textwidth]{../figs/figs_v11/ELECTRON_WGamma/PrepareYields/c_TotalDATAvsMC_EtaCommon__WMtVERY_PRELIMINARY.pdf}
  \caption{ $M_T^W$ of $W\gamma$ candidates. Data vs MC plots. Left: muon channel, right: electron channel. All selection criteria except $M_{T}^W$ requirement are applied on these plots. $15$~GeV$<P_T^{\gamma}<45$~GeV. }
  \label{fig:DATAvsMC_WMt}
  \end{center}
\end{figure}

\begin{figure}[htb]
  \begin{center}
   \includegraphics[width=0.7\textwidth]{../figs/figs_v11/ELECTRON_WGamma/PrepareYields/c_TotalDATAvsMC_EtaCommon__Mpholep1PRELIMINARY_FOR_E_TO_GAMMA_WITH_PSV_CUT_pt15to500_.pdf}
  \caption{$M_{l\gamma}$ of $W\gamma$ candidates in the electron channel. Data vs MC plots. All selection criteria except $Z$-mass window are applied on this plot. $P_T^{\gamma}>15$~GeV. }
  \label{fig:DATAvsMC_Mpholep1}
  \end{center}
\end{figure}

\begin{figure}[htb]
  \begin{center}
   \includegraphics[width=0.45\textwidth]{../figs/figs_v11/MUON_WGamma/PrepareYields/c_TotalDATAvsMC_Barrel__phoEt.pdf}\includegraphics[width=0.45\textwidth]{../figs/figs_v11/ELECTRON_WGamma/PrepareYields/c_TotalDATAvsMC_Barrel__phoEt.pdf}
   \includegraphics[width=0.45\textwidth]{../figs/figs_v11/MUON_WGamma/PrepareYields/c_TotalDATAvsMC_Endcap__phoEt.pdf}\includegraphics[width=0.45\textwidth]{../figs/figs_v11/ELECTRON_WGamma/PrepareYields/c_TotalDATAvsMC_Endcap__phoEt.pdf}
  \caption{$P_T^{\gamma}$ of $W\gamma$ candidates. Data vs MC plots. Left column: muon channel, right column: electron channel. Top to bottom: photons in EB and EE of ECal.}
  \label{fig:DATAvsMC}
  \end{center}
\end{figure}



%\begin{figure}[htb]
%  \begin{center}
%   \includegraphics[width=0.45\textwidth]{figs_v8/MUON_WGamma/PrepareYields/c_TotalDATAvsMC_Barrel__WMtVERY_PRELIMINARY.pdf}\includegraphics[width=0.45\textwidth]{figs_v8/ELECTRON_WGamma/PrepareYields/c_TotalDATAvsMC_Barrel__WMtVERY_PRELIMINARY.pdf}
%   \includegraphics[width=0.45\textwidth]{figs_v8/MUON_WGamma/PrepareYields/c_TotalDATAvsMC_Endcap__WMtVERY_PRELIMINARY.pdf}\includegraphics[width=0.45\textwidth]{figs_v8/ELECTRON_WGamma/PrepareYields/c_TotalDATAvsMC_Endcap__WMtVERY_PRELIMINARY.pdf}
%  \caption{Data vs MC plots, $M_T^W$. Left column - muon channel, right column - electron. Top to bottom: barrel and endcap photons. All selection criteria except $M_T^W$ cut on these plots. $15$~GeV$<P_T^{\gamma}<45$~GeV. The analysis cut of $M_T^W>40$~GeV is selected.}
%  \label{fig:DATAvsMC_WMt}
%  \end{center}
%\end{figure}
