\section{Object and Event selection}
\label{sec:ObjectSelection}
In this Section we document the electron, muon,
and photon identification and isolation criteria, MET criteria, and provide the
results of comparing MC simulation with data.
%We use cut-based selection of lepton + photon + MET where lepton is either muon or electron. 
%Photon and MET selection are a little bit different for different channels. 
%The second lepton veto is applied.  dR(lep,pho)$>$0.7 where $dR=\sqrt{({d\phi}^2+{d\eta}^2)}$ 
%is applied to avoid divergence of ISR and FSR contributions. No restrictions applied on a number of
% photons in the event but the candidate with the hardest photon is selected in the particular event among 
%those photons which passed all the other cuts including dR. dR is also a part of the phase space selection.\\
%The plot of total data vs MC after the selection criteria applied is shown in Fig. \ref{fig:DATAvsMC}.

\subsection{Object selection}
The muon selection includes the kinematics cuts $p_T>25$ GeV and $|\eta|<2.1$ and "Tight" muon identification and isolation
selections as recommended by POG~\cite{muPOG}. 
If there is the second reconstructed muon candidate with $P_T>10$ GeV and $|\eta|<2.4$ in the event, then the whole event is vetoed. 
%No muon ID requirements on the muon to be vetoed are applied.\\
Data to MC scale factors are applied as recommended by POG~\cite{SFmuPOG}.\\

We consider electrons with $p_T>30$~GeV and passing
the "Tight" identification and isolation selections as optimized by EGamma-POG for 2012 analysis~\cite{egmPOG}.
%We summarize electron identification and isolation requirements in Table%~\ref{}.
The ECAL fiducial region is defined in terms of barrel and endcap
sections with pseudorapidity ranges of $|\eta| < 1.4442$ and  
$1.566 < |\eta| < 2.5$, respectively. An electron is considered
to be within this ECAL acceptance if its associated SuperCluster (SC) is
within the ECAL acceptance.
 If there is the second reconstructed electron candidate with $p_T>10$ GeV and satisfying the "VETO" identification and isolation selections~\cite{egmPOG}
in the event, then the whole event is vetoed.
Data to MC scale factors are applied as recommended by POG~\cite{SFegmPOG}.

Photon candidates are reconstructed as SuperClusters with 
$p_{T} > 15$~GeV in the fiducial volume of the ECAL detector:
barrel (EB) with $|\eta|<1.4442$ and endcap (EE) with 
$1.566 < |\eta| < 2.5$. To reduce copious background objects from jets misidentified as photons 
we apply the "Medium" identification and isolation selections as recommended by EGamma-POG~\cite{phoPOG}.
Data to MC scale factors are applied as recommended by POG~\cite{SFphoPOG}.
%Phosphor correction is not applied.\\
PixelSeedVeto instead of ElectronConversionSafeVeto applied for the electron channel. Scale factors for PixelSeedVeto are taken from the W$\gamma\gamma$ analysis~\cite{Wgg8TeV}.
dR(lep,pho)$>$0.7 where $dR=\sqrt{({d\phi}^2+{d\eta}^2)}$ 
is applied to avoid divergence of ISR and FSR contributions. No restrictions applied on a number of
 photons in the event but the candidate with the hardest photon is selected in the particular event among 
those photons which passed all the other cuts including dR. dR is also a part of the phase space selection.\\

\subsection{Event Level Selection}
W transverse mass cut is applied of $M_T^W>40$~GeV where $M_T=\sqrt{(2 \cdot P_T^{Lep} \cdot P_T^{MET} \cdot (1-\cos{(\phi^{lep}-\phi^{MET})}))}$. MET (missing transverse energy) reconstructed with particle flow algorithm was used~\cite{pfMET}. \\

Z-mass window cut is applied in electron channel to suppress DYjets background: events which fall into the Z-mass window $70$~GeV$<M_{e\gamma}<110$~GeV are rejected.\\

\begin{figure}[htb]
  \begin{center}
   \includegraphics[width=0.5\textwidth]{../figs/figs_v11/MUON_WGamma/PrepareYields/c_TotalDATAvsMC_EtaCommon__Mpholep1_pt15to500_.pdf}\includegraphics[width=0.5\textwidth]{../figs/figs_v11/ELECTRON_WGamma/PrepareYields/c_TotalDATAvsMC_EtaCommon__Mpholep1PRELIMINARY_FOR_E_TO_GAMMA_WITH_PSV_CUT_pt15to500_.pdf}
  \caption{Data vs MC plots, $M_{l,\gamma}$. Left - muon channel, right - electron. All selection criteria except $M_{l,\gamma}$ cut on these plots. $<P_T^{\gamma}>15$~GeV. The analysis cut of $M_{l,\gamma}<70~or~M_{l,\gamma}>110$~GeV is selected for the electron channel only.}
  \label{fig:DATAvsMC_Mpholep1}
  \end{center}
\end{figure}

\subsection{Weights and Corrections}
The pile up reweighting is applied. Fig. \ref{fig:DATAvsMC_nVtx} shows the distribution of the number of vertices for the Z$\gamma$ selected sample in muon channel before (left) and after (right) the pile up reweighting of the MC samples. The same procedure of the pile up reweighting is applied for the W$\gamma$ selected MC samples.
\begin{figure}[htb]
  \begin{center}
   \includegraphics[width=0.45\textwidth]{../figs/figs_v11/MUON_ZGamma/PrepareYields/c_TotalDATAvsMC_EtaCommon__nVtx_noPU.png}\includegraphics[width=0.45\textwidth]{../figs/figs_v11/MUON_ZGamma/PrepareYields/c_TotalDATAvsMC_EtaCommon__nVtx.png}
  \caption{Plots ofnumber of vertices, Data vs MC plots. Z$\gamma$ selected sample, muon channel. Left: no PU reweighting applied, right: PU reweighting applied. }
  \label{fig:DATAvsMC_nVtx}
  \end{center}
\end{figure}

This is a policy of the CMS SMP group that all the analysis must be performed in a blinded way. We discussed how to implement blinding in our case with the statistics committee [REFERENCE] and were given the following recipe:\\
\begin{itemize}
  \item use all data for $p_T^{\gamma}<p_T^{\gamma-threshold}$, where $p_T^{\gamma-threshold}$ is the value after which the analysis becomes potentially sensitive to the new physics; and
  \item for $p_T^{\gamma}>p_T^{\gamma-threshold}$, use $X\%$ of data and pick  $X\%$ in such a way that $err_{stat} \geq 3 \cdot err_{syst}$.
\end{itemize}

We decided on $p_T^{\gamma-threshold}=45~GeV$ and, relying on systematic error of ~10\%, quoted by 7 TeV measurement [REFERENCE], on X\%=5\%. Therefore, our blinding strategy is:
\begin{itemize}
  \item for $p_T^{\gamma}<45~GeV$: use full data; and
  \item for $p_T^{\gamma}>45~GeV$: use $5\%$ of data.
\end{itemize}

Our Wjets, DYjets and $t\bar{t}jets$ MC samples partially contain $W\gamma$, $Z\gamma$, $t\bar{t}\gamma$ events. To remove the overlap, we did the following, relying totally on the MC-truth information available in Wjets, DYjets, $t\bar{t}jets$ samples:
\begin{itemize}
  \item loop over all gen-level photons available in the ggNtuple;
  \item the event is removed from selected sample if at least one photon satisfy the following criteria:
  \begin{itemize}
     \item has $P_T^{\gamma}>10~GeV$;
     \item originates from lepton, boson or quark based on it's parentage information; and
     \item is separated by $dR(pho,lep)>0.4$ from at least one gen-muon or gen-electron, depending on the channel.
  \end{itemize}
\end{itemize}

A good data vs MC agreement for the Z$\gamma$ plots in Fig. \ref{fig:DATAvsMC_nVtx} validates the procedure.

\subsection{Selected Events}

%Distributions of $P_T^{\gamma}$ and $M_T^W$ of the selected events are shown in Fig. \ref{fig:DATAvsMC} and \ref{fig:DATAvsMC_WMt}. 
%The is a large discrepancies in all the distributions and therefore the data-driven background estimates are necessary.\\

Distributions of $P_T^{\gamma}$ of the selected events are shown in Fig.~\ref{fig:DATAvsMC}. The is a large discrepancies in all the distributions and therefore the data-driven background estimates are necessary.

\begin{figure}[htb]
  \begin{center}
   \includegraphics[width=0.45\textwidth]{../figs/figs_v11/MUON_WGamma/PrepareYields/c_TotalDATAvsMC_Barrel__phoEt.pdf}\includegraphics[width=0.45\textwidth]{../figs/figs_v11/ELECTRON_WGamma/PrepareYields/c_TotalDATAvsMC_Barrel__phoEt.pdf}
   \includegraphics[width=0.45\textwidth]{../figs/figs_v11/MUON_WGamma/PrepareYields/c_TotalDATAvsMC_Endcap__phoEt.pdf}\includegraphics[width=0.45\textwidth]{../figs/figs_v11/ELECTRON_WGamma/PrepareYields/c_TotalDATAvsMC_Endcap__phoEt.pdf}
  \caption{Data vs MC plots. Left column - muon channel, right column - electron. Top to bottom: barrel and endcap photons}
  \label{fig:DATAvsMC}
  \end{center}
\end{figure}



%\begin{figure}[htb]
%  \begin{center}
%   \includegraphics[width=0.45\textwidth]{figs_v8/MUON_WGamma/PrepareYields/c_TotalDATAvsMC_Barrel__WMtVERY_PRELIMINARY.pdf}\includegraphics[width=0.45\textwidth]{figs_v8/ELECTRON_WGamma/PrepareYields/c_TotalDATAvsMC_Barrel__WMtVERY_PRELIMINARY.pdf}
%   \includegraphics[width=0.45\textwidth]{figs_v8/MUON_WGamma/PrepareYields/c_TotalDATAvsMC_Endcap__WMtVERY_PRELIMINARY.pdf}\includegraphics[width=0.45\textwidth]{figs_v8/ELECTRON_WGamma/PrepareYields/c_TotalDATAvsMC_Endcap__WMtVERY_PRELIMINARY.pdf}
%  \caption{Data vs MC plots, $M_T^W$. Left column - muon channel, right column - electron. Top to bottom: barrel and endcap photons. All selection criteria except $M_T^W$ cut on these plots. $15$~GeV$<P_T^{\gamma}<45$~GeV. The analysis cut of $M_T^W>40$~GeV is selected.}
%  \label{fig:DATAvsMC_WMt}
%  \end{center}
%\end{figure}
