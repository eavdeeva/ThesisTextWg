\subsection{Electroweak Theory of the Standard Model}
\label{sec:WgAbout_SMEWK}

%Pich, 1

The SM is a gauge theory invariant under the local $SU(3)_C x SU(2)_L x U(1)_Y$ transformation. $SU(3)_C$ stands for color transformations and, therefore, the $SU(3)_C$-invariant Lagrangian describes QCD. It is designed to describe color interactions of quarks and antiquarks. The requirement to satisfy the gauge invariance generates eight massless gluons. The non-abelian nature of the $SU(3)$ group leads to self-interactions of gluons, particularly to the appearance of three-gluon and four-gluon vertices.\\

The Lagrangian based on the $SU(2)_L x U(1)_Y$ symmetry describes unified theory of electroweak interactions. The requirement of the local gauge invariant generates four massless vector bosons which are mediators of electromagnetic and weak interactions. The non-abelian structure of $SU(2)$ group introduces gauge boson couplings just like self-interactions of gluons appear in QCD. However, it is experimentally known that mediators of weak interactions are heavy particles with masses $M_W=80$~GeV and $M_Z=91$~GeV. To introduce these masses, the electroweak symmetry has to be spontaneously broken to the QED symmetry group $U(1)$:\\

$SU(2)_L \times U(1)_Y \rightarrow U(1)_{QED}$\\

The Spontaneous Symmetry Breaking (SSB) introduces an additional particle into the SM: the Higgs boson ($H$) and generates masses for $W$ and $Z$ bosons. A photon is a massless particle, therefore, $U(1)_{QED}$ symmetry remains unbroken and a photon does not couple to a Higgs boson.\\

The Lagrangian transformations of the SM are described in \cite{ref_Pich} for QED, QCD, unified electroweak force and the Higgs mechanis. The measurement in this dissertation provides a test for the electroweak sector of the SM, thus we will repeat here the theoretical path from the Lagrangians of free particles to the accomodation of the electroweak gauge bosons including their self-couplings.\\

%Pich, 3.1
% introduce hirality/helicity
There are certain features of weak interactions whach are known experimentally and they all have to be accomodated by the theory. It is known that only left-handed fermions and only right-handed antifermions couple to a $W$ boson.\\ 

A $Z$ boson couples to both left-handed and right-handed fermions and antifermions but with different strength while photon-fermion couplings do not depend on chirality but on electric charge only. As for the neutrinos, only left-handed neutrinos and right-handed antineutrinos found to couple to a $Z$ boson, therefore, right-handed neutrinos and left-handed antineutrinos were not found to participate in any SM interactions.\\

Given different properties of left-handed and right-handed particles, they are treated differently by the electroweak theory. $SU(2)$ doublets are introduced for the wave functions of left-handed particles while $SU(2)$ singlets are introduced for the wave functions of right-handed particles.\\ 
 
%Pich 3.2

\begin{equation}\label{eq:psi_for_quarks}
\psi_1(x)=\begin{pmatrix} u \\ d \end{pmatrix}_L \text{, } \psi_2(x)=u_R \text{, } \psi_3(x)=d_R \text{.}
\end{equation}

\begin{equation}\label{eq:psi_for_leptons}
\psi_1(x)=\begin{pmatrix} \nu_e \\ e^- \end{pmatrix}_L \text{, } \psi_2(x)=\nu_{eR} \text{, } \psi_3(x)=e^-_R \text{. }
\end{equation}

Consider the free Lagrangian:\\

\begin{equation}\label{eq:L_free}
L_0 = \sum_{j=1}^{3} i \bar{\psi_j}(x) \gamma^\mu \partial_\mu \psi_j(x) 
\end{equation}

where $\gamma^\mu$ are Dirac matrices \cite{ref_Griffiths}.\\

The wave function $\psi_1$ changes under the $SU(2)_L \times U(1)_Y$ transformations in the following way:\\

\begin{equation}
\psi_1(x) \rightarrow e^{i y_1 \beta} U_L \psi_1(x)
\end{equation}


The wave functions $\psi_{(2,3)}(x)$ are singlets of $SU(2)_L$ and are affected only by $U(1)$ transformations:\\

\begin{equation}
\psi_{(2,3)}(x) \rightarrow e^{i y_{(2,3)} \beta} \psi_{(2,3)}(x)
\end{equation}

The transformation $U_L \equiv e^{i \sigma_i \alpha_i /2}$, $\sigma_i$ are Pauli matrices \cite{ref_Griffiths}, $\alpha_i(x)$ and $\beta(x)$ are arbitrary functions, $y_{(1,2,3)}$ are hypercharges.\\

In order to satisfy the local Largangian invariance, partial derivatives in \ref{eq:L_free} has to be substituted with covariant derivatives:\\

\begin{equation}\label{eq:L_free_covariant}
L_0 \rightarrow L = \sum_{j=1}^{3} i \bar{\psi_j}(x) \gamma^\mu D_\mu \psi_j(x) 
\end{equation}

where\\

\begin{equation}
D_\mu \psi_1(x) = [\partial_\mu - i g {\tilde{W}}_\mu(x) - i g' y_1 B_\mu(x) ] \psi_1(x) 
\end{equation}

\begin{equation}
D_\mu \psi_{(2,3)}(x) = [\partial_\mu - i g' y_{(2,3)} B_\mu(x) ] \psi_{(2,3)}(x) 
\end{equation}

where ${\tilde{W}}_\mu(x) \equiv \frac{\sigma_i}{2} W_\mu^i(x) = \frac{1}{\sqrt{2}} 
\begin{pmatrix}
\sqrt{2} W_\mu^3 & (W_\mu^1 - i W_\mu^2)/{\sqrt{2}}\\
(W_\mu^1 + i W_\mu^2)/{\sqrt{2}} & -W_\mu^3\\
\end{pmatrix}$.

The Lagrangian $L$ from Eq.~\ref{eq:L_free_covariant} is now invariant under local $SU(2)_L \times U(1)$ transformations. Four vector boson fields appear in $L$: $B_\mu$, $W_\mu^1$, $W_\mu^2$, $W_\mu^3$.  Thus, it is necessary to add terms for kinetic energies of the vector bosons:\\

\begin{equation} \label{eq:L_gauge_kin}
L_{KIN}=-\frac{1}{4}B_{\mu\nu}B^{\mu\nu}-\frac{1}{4}W_{\mu\nu}^i W^{\mu\nu}_i
\end{equation}

where $B_{\mu\nu} \equiv \partial_\mu B_\nu - \partial_\nu B_\mu$, $W_{\mu\nu}^i \equiv \partial_\mu W_\nu^i - \partial_\nu W_\mu^i + g \epsilon^{ijk} W_\mu^j W_\nu^k$\\

Off-diagonal terms of ${\tilde{W}}_\mu$ are wave functions of charged vector bosons $W^{\pm}=W_\mu^1 \mp i W_\mu^2)/{\sqrt{2}}$ while $W_\mu^3$ and $B_\mu$ are neutral fields corresponding to a $Z$ boson and a photon. However, $W_\mu^3$ couples to left-handed fermions only while a $Z$ boson is known to interact with particles with both helicities.\\

Then the neutral electroweak mixing is introduced:\\

\begin{equation}
  \begin{pmatrix} W_\mu^3 \\ B_\mu \end{pmatrix} \equiv
  \begin{pmatrix} \cos \theta_W & \sin \theta_W \\ -\sin \theta_W & \cos \theta_W \end{pmatrix}
  \begin{pmatrix} Z_\mu \\ A_\mu \end{pmatrix}
\end{equation} 

where $\theta_W$ is an electroweak mixing angle, $A_\mu$ is a photon field.\\

Terms involving $A_\mu$ in the electroweak Largangian must be equal to the corresponding terms in QED Lagrangian:\\

\begin{equation}\label{eq:L_QED}
L_{QED} = i \bar{\psi}(x) \gamma^\mu \partial_\mu \psi(x) - m \bar{\psi}(x) \psi(x) + e Q A_\mu(x) \bar{\psi}(x) \gamma^\mu \psi(x) - \frac{1}{4} A_\mu\nu(x) A^{\mu\nu}(x)
\end{equation}

This requirement relates $g$, $g'$, $\theta_W$ and $e$ as\\

$g \sin \theta_W = g' \cos \theta_W = e$\\

and provides expression for weak hypercharges:\\

$Y = Q - T_3$,\\

where $T_3 = \sigma_3 / 2$, $Q_1 = \begin{pmatrix} Q_{u/\nu} & 0 \\ 0 & Q_{d/e} \end{pmatrix}$, $Q_2 = Q_{u/\nu}$, $Q_3=Q_{d/e}$.\\

If substitute $W_\mu^i$ and $B_\mu$ in $L$ from Eq. \ref{L_gauge_kin} with the wave functions of $W^\pm$, $Z$ and a photon, charged TGC and QGC terms will apeear as shown in Eq.~\ref{eq:L_TGC_1} and Eq.~\ref{eq:L_QGC_1} but not neutral TGC or QGC vertex would be generated.\\

\begin{equation} \label{eq:EWK_Zg_bosons_mixing}
B_\mu = -sin \theta_W Z_\mu + cos \theta_W A_\mu \text{, } W_\mu^3 = cos \theta_W Z_\mu + sin \theta_W A_\mu
\end{equation}
\begin{equation} \label{eq:EWK_Zg_bosons_mixing}
W_\mu^1 = \sqrt{2}(W^+ + W^-) \text{, }W_\mu^2 = \sqrt{2}(W^- + W^+)
\end{equation}

\begin{equation} \label{eq:L_TGC_1}
L_{TGC} = -\frac{g}{4}(\partial_\mu W_\nu^i - \partial_\nu W_\mu^i)\epsilon^{ijk}W^{\mu j}W^{\nu k} - \frac{g}{4}\epsilon^{ijk}W_\mu^j W_\nu^k (\partial^\mu W^{\nu i} - \partial^\nu W^{\mu i})
\end{equation}

\begin{equation} \label{eq:L_QGC_1}
L_{QGC} = -\frac{g^2}{4} \epsilon^{ijk} \epsilon^{ilm} W_\mu^j W_\nu^k W^{\mu l} W^{\nu m}
\end{equation}

\begin{equation} \label{eq:L_TGC_2}
L_{TGC} = L_{TGC}^{(1)} + L_{TGC}^{(2)}
\end{equation}

\begin{equation} \label{eq:L_TGC_2_1}
L_{TGC}^{(1)} = -ie \cot \theta_W (W^{-\mu\nu} W^{+}_\mu Z_\nu - W^{+\mu\nu} W^-_\mu Z_\nu +W^-_\mu W^+_\nu Z^{\mu\nu}) 
\end{equation}

\begin{equation} \label{eq:L_TGC_2_2}
L_{TGC}^{(2)} = - ie(W^{-\mu\nu}W^+_\mu A_\nu - W^{+\mu\nu}W^-_\mu A_\nu + W^-_\mu W^+_\nu A^{\mu\nu})
\end{equation}

\begin{equation} \label{eq:L_QGC_2}
L_{QGC} = L_{QGC}^{(1)} + L_{QGC}^{(2)} + L_{QGC}^{(3)} + L_{QGC}^{(4)}
\end{equation}

\begin{equation} \label{eq:L_QGC_2_1}
L_{QGC}^{(1)} = -\frac{e^2}{2\sin^2 \theta_W}(W^+_\mu W^{-\mu}W^+_\nu W^{-\nu} - W^+_\mu W^{\mu +} W^-_\nu W^{-\nu})
\end{equation}

\begin{equation} \label{eq:L_QGC_2_2}
L_{QGC}^{(2)} = - e^2 \cot^2 \theta_W (W^+_\mu W^{-\mu} Z_\nu Z^{\nu} - W^+_\mu Z^{\mu} W^-_\nu Z^{\nu})
\end{equation}

\begin{equation} \label{eq:L_QGC_2_3}
L_{QGC}^{(3)} = - e^2 \cot \theta_W (2 W_\mu^+ W^{-\mu} Z_\nu A^{\nu} - W^{+}_\mu Z^\mu W^-_\nu A^\nu - W^{+}_\mu A^\mu W^-_\nu Z^\nu)
\end{equation}

\begin{equation} \label{eq:L_QGC_2_4}
L_{QGC}^{(4)} = - e^2 (W^+_\mu W^{-\mu} A_\nu A^{\nu} - W^+_\mu A^{\mu} W^-_\nu A^{\nu})
\end{equation}

Because left-handed and right-handed wave functions transform differently under the electroweak symmetry, fermion mass terms would not be invariant and, therefore, are forbidden. Mass terms for gauge bosons also would violate the Lagrangian invariance. Therefore, Largangian $L$ in Eq. \ref{eq:L_free_covariant} contains massless particles only.\\

To introduce masses into the electroweak Lagrangian, an $SU(2)_L$ doublet of complex scalar fields $\phi(x)$ is added to the Lagrangian (Eq.~\ref{eq:H_doublet}).\\

\begin{equation}\label{eq:H_doublet}
  \phi(x) \equiv \begin{pmatrix} \phi^{(+)}(x) \\ \phi^{(0)}(x) \end{pmatrix}
\end{equation}

One of the components of the doublet involves a new particles, the Higgs boson \cite{ref_Pich}. Masses of $W$ and $Z$ bosons and charged fermions aquire their masses through couplings to the Higgs boson. The strength of the coupling constant is proportional to the square of the particle's mass, therefore, massless particles do not couple to $H$.\\

The mechanism of generating a fermion's mass involve both left-handed and right-handed components of the fermion. If our hypothesis that right-handed neutrinos do not exist, then the Higgs mechanism does not generate neutrino masses. However, from the experiments of neutrino oscillations, neutrinos are known to have masses even though their masses are much smaller than masses of other fermions. Several hypotheses were offered to resolve this contradiction however at the moment the mechanism of neutrinos to acquire masses remain unknown \cite{ref_PDG}.\\

