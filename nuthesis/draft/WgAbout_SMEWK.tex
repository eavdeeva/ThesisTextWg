\subsection{Electroweak Theory of the Standard Model}
\label{sec:WgAbout_SMEWK}

%Pich, 1
To develop a quantum field theory, we start with the Lagrangian of free fermions. In order to describe a system with a conservation of a physical quantity, the Lagrangian is required to satisfy a local invariance with respect to a certain transformation. For instance, a conservation of an electric charge requires a local invariance under $U(1)$ transformation for the QED Lagrangian~\cite{ref_Pich}. The requirement of the local invariance introduces an interaction of a new vector field (or several fields) with our free fermions. The new vector field is a mediator of an interaction conserving the physical quantity. To provide a full description for a new boson field, in addition to the interaction term we introduce an invariant term for the kinetic energy of the boson. Such approach allows us to derive the Lagrangian which is locally invariant with respect to a certain gauge transformation and contains interacting fermions as well as interaction mediators. \\ 

The SM is a quantum field theory invariant under the local $SU(3)_C \times SU(2)_L \times U(1)_Y$ transformation \cite{ref_Pich}. The SM Lagrangian includes all observed quantum fields and their interactions. \\ 

The part of the SM Lagrangian based on the $SU(3)_C$ symmetry and is called QCD or theory of strong interactions. QCD has three types of charges which are called colors: red, blue, and green. To be a subject of strong interaction, a fermion must posses a color charge. Quarks and antiquarks are such fermions. The requirement to satisfy the gauge invariance with respect to $SU(3)_C$ transformations generates eight massless gluons, and the non-abelian nature of the $SU(3)$ group generates self-interactions of gluons including three-gluon and four-gluon vertices.\\

The part of the SM Lagrangian based on the $SU(2)_L \times U(1)_Y$ symmetry is a foundation of the unified theory of electroweak interactions. $SU(2)_L$ reflects transformations in the weak isospin space of left-handed fermions (\cite{ref_Griffiths},~Ch.~9) while $U(1)_Y$ reflects transformations in a weak hypercharge space of all fermions. The requirement of the local gauge invariant generates four massless vector bosons which are mediators of electromagnetic and weak interactions. The non-abelian structure of $SU(2)$ group generates gauge boson couplings the same way as self-interactions of gluons appear in QCD.\\ 

Mass terms for the vector bosons would violate the gauge invariance of the electroweak Lagrangian, however it is experimentally known that mediators of weak interactions are heavy particles with masses $M_W=80$~GeV and $M_Z=91$~GeV. A possible solution of the discrepancy is a mechanism of the spontateous symmetry breaking. QED symmetry group $U(1)$ remains unbroken because a photon is massless.\\

% REPHRASE ENTIRELY
The mechanism of the Spontaneous Symmetry Breaking and the appearance of the mass terms for $W$ and $Z$ boson is realized by introducing an additional doublet of scalar fields. After that, the Lagrangian is being transformed in such a way that $W$ and $Z$ bosons acquire masses through their interactions with a new particle: a Higgs boson ($H$). A photon does not couple to the Higgs boson remaining a massless particle and keeping $U(1)_{QED}$ symmetry unbroken.\\

The measurement in this dissertation provides a test for the electroweak sector of the SM. We will retrace the steps of the derivation of the EWK part of the SM Lagrangian starting from terms of free fermions. The resulting Lagrangian accommodates electroweak gauge bosons including their self-couplings. One of these self-couplings, $WW\gamma$, is the primary focus of our measurement.\\

%Pich, 3.1
It is experimentally known that dynamics of weak interactions depends on particle's chirality (\cite{ref_Griffiths},~Ch.~4.4.1). In particular, a $W$ boson couples to left-handed fermions and right-handed antifermions only. A $Z$ boson couples to both left-handed and right-handed charged fermions and antifermions but only to left-handed neutrinos and right-handed antineutrinos. Given different properties of left-handed and right-handed fermions, they are treated differently by the electroweak theory. $SU(2)$ doublets are introduced for the wave functions of left-handed fermions while $SU(2)$ singlets are introduced for the wave functions of right-handed fermions. Equations~\ref{eq:psi_for_quarks}~and~\ref{eq:psi_for_leptons} show wave functions for the first generation fermions. Wave functions for the other two generations are constructed the same way.\\ 
 
%Pich 3.2
\begin{equation}\label{eq:psi_for_quarks}
\psi_1(x)=\begin{pmatrix} u \\ d' \end{pmatrix}_L \text{, } \psi_2(x)=u_R \text{, } \psi_3(x)=d'_R \text{.}
\end{equation}

\begin{equation}\label{eq:psi_for_leptons}
\psi_1(x)=\begin{pmatrix} \nu_e \\ e^- \end{pmatrix}_L \text{, } \psi_2(x)=\nu_{eR} \text{, } \psi_3(x)=e^-_R \text{. }
\end{equation}

\noindent{The state $d'$ in Eq.~\ref{eq:psi_for_quarks} is a mixture of $d$, $c$ and $b$ quark's wave functions and is determined by the quark mixing matrix which is also called Cabbibo-Kobayashi-Maskawa matrix \cite{ref_Pich}:}\\

\begin{equation}
  \begin{pmatrix} d' \\ c' \\ b' \end{pmatrix} = V
  \begin{pmatrix} d \\ c \\ b \end{pmatrix}
\end{equation}

To derive the unified electroweak Lagrangian, we start with the free fermion terms:\\

\begin{equation}\label{eq:L_free}
L_0 = \sum_{j=1}^{3} i \bar{\psi_j}(x) \gamma^\mu \partial_\mu \psi_j(x), 
\end{equation}

\noindent{where $\gamma^\mu$ are Dirac matrices (\cite{ref_Griffiths},~Ch.~7.1) and $\psi_j(x)$ are wave functions determined by Eqs.~\ref{eq:psi_for_quarks}~and~\ref{eq:psi_for_leptons}.}\\

The wave function $\psi_1$ changes under the $SU(2)_L \times U(1)_Y$ transformations in the following way:\\

\begin{equation}\label{eq:psi1_transform}
\psi_1(x) \rightarrow e^{i y_1 \beta} U_L \psi_1(x),
\end{equation}

\noindent{while the wave functions $\psi_{(2,3)}(x)$ are singlets of $SU(2)_L$ and are affected only by $U(1)$ transformations:}\\

\begin{equation}{eq:psi23_transform}
\psi_{(2,3)}(x) \rightarrow e^{i y_{(2,3)} \beta} \psi_{(2,3)}(x).
\end{equation}

\noindent{The transformation in the weak isospin space is defined as $U_L \equiv e^{i \sigma_i \alpha_i /2}$ where $\sigma_i$ are Pauli matrices~(\cite{ref_Griffiths},~Ch.~4.2.2). Phases  $\alpha_i(x)$ and $\beta(x)$ in Eqs.~\ref{eq:psi1_transform}~and~\ref{eq:psi23_transform} are arbitrary functions of $x$, and $y_{(1,2,3)}$ are weak hypercharges which are named analogous to electric charges in QED.}\\

In order to satisfy the local $SU(2)_L \times U(1)_Y$ invariance, partial derivatives in Eq.~\ref{eq:L_free} have to be substituted with covariant derivatives:\\

\begin{equation}
D_\mu \psi_1(x) = [\partial_\mu - i g {\tilde{W}}_\mu(x) - i g' y_1 B_\mu(x) ] \psi_1(x) 
\end{equation}

\begin{equation}
D_\mu \psi_{(2,3)}(x) = [\partial_\mu - i g' y_{(2,3)} B_\mu(x) ] \psi_{(2,3)}(x) 
\end{equation}

\noindent{where $g$, $g'$ are arbitrary constants,}\\ 

\begin{equation}
  {\tilde{W}}_\mu(x) \equiv \frac{\sigma_i}{2} W_\mu^i(x) = \frac{1}{\sqrt{2}} 
  \begin{pmatrix}
  \sqrt{2} W_\mu^3 & (W_\mu^1 - i W_\mu^2)/{\sqrt{2}}\\
  (W_\mu^1 + i W_\mu^2)/{\sqrt{2}} & -W_\mu^3\\
  \end{pmatrix} , 
\end{equation}

\noindent{$B_\mu$, $W_\mu^1$, $W_\mu^2$, $W_\mu^3$ are four vector bosons that arise from the requirement of the Lagrangian to be invariant under local $SU(2)_L \times U(1)$ transformations.}\\

The Lagrangian becomes:\\

\begin{equation}\label{eq:L_free_covariant}
L_0 \rightarrow L = \sum_{j=1}^{3} i \bar{\psi_j}(x) \gamma^\mu D_\mu \psi_j(x) 
\end{equation}

\noindent{To make new vector bosons physical fields it is necessary to add terms for their kinetic energies:}\\

\begin{equation} \label{eq:L_gauge_kin}
L_{KIN}=-\frac{1}{4}B_{\mu\nu}B^{\mu\nu}-\frac{1}{4}W_{\mu\nu}^i W^{\mu\nu}_i
\end{equation}

\noindent{where $B_{\mu\nu} \equiv \partial_\mu B_\nu - \partial_\nu B_\mu$, $W_{\mu\nu}^i \equiv \partial_\mu W_\nu^i - \partial_\nu W_\mu^i + g \epsilon^{ijk} W_\mu^j W_\nu^k$}\\

Off-diagonal terms of ${\tilde{W}}_\mu$ are wave functions of charged vector bosons $W^{\pm}=(W_\mu^1 \mp i W_\mu^2)/{\sqrt{2}}$ while $W_\mu^3$ and $B_\mu$ are neutral fields which are mixtures of a $Z$ boson and a photon determined by: \\

\begin{equation}
  \begin{pmatrix} W_\mu^3 \\ B_\mu \end{pmatrix} \equiv
  \begin{pmatrix} \cos \theta_W & \sin \theta_W \\ -\sin \theta_W & \cos \theta_W \end{pmatrix}
  \begin{pmatrix} Z_\mu \\ A_\mu \end{pmatrix}
\end{equation} 

\noindent{where $\theta_W$ is an electroweak mixing angle, $A_\mu$ is a photon field.}\\

In order to be consistent with QED, terms involving $A_\mu$ in the electroweak Lagrangian must be equal to the corresponding terms in QED Lagrangian~\cite{ref_Pich}:\\

\begin{equation}\label{eq:L_QED}
L_{QED} = i \bar{\psi}(x) \gamma^\mu \partial_\mu \psi(x) - m \bar{\psi}(x) \psi(x) + q A_\mu(x) \bar{\psi}(x) \gamma^\mu \psi(x) - \frac{1}{4} F_{\mu\nu}(x) F^{\mu\nu}(x),
\end{equation}

\noindent{where $q$ is electric charge of $\psi(x)$ field, $F_{\mu\nu} \equiv \partial_\mu A_\nu - \partial_\nu A_\mu$.}\\

This requirement relates $g$, $g'$, $\theta_W$ and $e$ as $g \sin \theta_W = g' \cos \theta_W = e$ and provides expression for weak hypercharges: $y = q - t_3$, where $q$ is the electric charge and $t_3$ is a $z$-component of the weak isospin. This results in $y_1=1/6$, $y_2=2/3$, and $y_3=-1/3$ for quarks and $y_1=-1/2$, $y_2=0$, and $y_3=-1$ for leptons. A right-handed neutrino has a weak hypercharge of $y_2=0$. It also does not have an electric charge and, as a right-handed fermion, has $t_3=0$ and, therefore, does not couple to a $W$ boson. Thus, a right-handed neutrino does not participate in any SM interaction.\\

Writing $\tilde{W}_\mu$ in Eq.~\ref{eq:L_gauge_kin} explicitly, we obtain TGC and QGC coupling terms:\\ 

\begin{equation} \label{eq:L_TGC_1}
L_{TGC} = -\frac{g}{4}(\partial_\mu W_\nu^i - \partial_\nu W_\mu^i)\epsilon^{ijk}W^{\mu j}W^{\nu k} - \frac{g}{4}\epsilon^{ijk}W_\mu^j W_\nu^k (\partial^\mu W^{\nu i} - \partial^\nu W^{\mu i})
\end{equation}

\begin{equation} \label{eq:L_QGC_1}
L_{QGC} = -\frac{g^2}{4} \epsilon^{ijk} \epsilon^{ilm} W_\mu^j W_\nu^k W^{\mu l} W^{\nu m}
\end{equation}

Substituting $W_\mu^i$ and $B_\mu$ in Eq.~\ref{eq:L_TGC_1} and Eq.~\ref{eq:L_QGC_1} with the wave functions of $W^\pm$, $Z$ and a photon:\\

\begin{equation} \label{eq:EWK_Zg_bosons_mixing}
B_\mu = -\sin \theta_W Z_\mu + \cos \theta_W A_\mu \text{, } W_\mu^3 = \cos \theta_W Z_\mu + \sin \theta_W A_\mu,
\end{equation}
\begin{equation} \label{eq:EWK_Zg_bosons_mixing}
W_\mu^1 = \sqrt{2}(W^+ + W^-) \text{, }W_\mu^2 = \sqrt{2}(W^- + W^+),
\end{equation}

\noindent{we receive charged TGC and QGC Lagrangians in the forms of Eqs.~\ref{eq:L_TGC_2} and~\ref{eq:L_QGC_2}. }\\

Equation~\ref{eq:L_TGC_2} involves $WWZ$ (Eq.~\ref{eq:L_TGC_2_1}) and $WW\gamma$ (Eq.~\ref{eq:L_TGC_2_2}) interactions:\\

\begin{equation} \label{eq:L_TGC_2}
L_{TGC} = L_{TGC}^{(1)} + L_{TGC}^{(2)},
\end{equation}

\begin{equation} \label{eq:L_TGC_2_1}
L_{TGC}^{(1)} = -ie \cot \theta_W (W^{-\mu\nu} W^{+}_\mu Z_\nu - W^{+\mu\nu} W^-_\mu Z_\nu +W^-_\mu W^+_\nu Z^{\mu\nu}),
\end{equation}

\begin{equation} \label{eq:L_TGC_2_2}
L_{TGC}^{(2)} = - ie(W^{-\mu\nu}W^+_\mu A_\nu - W^{+\mu\nu}W^-_\mu A_\nu + W^-_\mu W^+_\nu A^{\mu\nu}).
\end{equation}

Equation~\ref{eq:L_QGC_2} involves $WWWW$ (Eq.~\ref{eq:L_QGC_2_1}), $WWZZ$ (Eq.~\ref{eq:L_QGC_2_2}), $WWZ\gamma$ (Eq.~\ref{eq:L_QGC_2_3}), and $WW\gamma\gamma$ (Eq.~\ref{eq:L_QGC_2_4}) interactions:\\

\begin{equation} \label{eq:L_QGC_2}
L_{QGC} = L_{QGC}^{(1)} + L_{QGC}^{(2)} + L_{QGC}^{(3)} + L_{QGC}^{(4)},
\end{equation}

\begin{equation} \label{eq:L_QGC_2_1}
L_{QGC}^{(1)} = -\frac{e^2}{2\sin^2 \theta_W}(W^+_\mu W^{-\mu}W^+_\nu W^{-\nu} - W^+_\mu W^{\mu +} W^-_\nu W^{-\nu}),
\end{equation}

\begin{equation} \label{eq:L_QGC_2_2}
L_{QGC}^{(2)} = - e^2 \cot^2 \theta_W (W^+_\mu W^{-\mu} Z_\nu Z^{\nu} - W^+_\mu Z^{\mu} W^-_\nu Z^{\nu}),
\end{equation}

\begin{equation} \label{eq:L_QGC_2_3}
L_{QGC}^{(3)} = - e^2 \cot \theta_W (2 W_\mu^+ W^{-\mu} Z_\nu A^{\nu} - W^{+}_\mu Z^\mu W^-_\nu A^\nu - W^{+}_\mu A^\mu W^-_\nu Z^\nu),
\end{equation}

\begin{equation} \label{eq:L_QGC_2_4}
L_{QGC}^{(4)} = - e^2 (W^+_\mu W^{-\mu} A_\nu A^{\nu} - W^+_\mu A^{\mu} W^-_\nu A^{\nu}).
\end{equation}

In the measurement of this dissertation we probe $WW\gamma$ coupling (Eq.~\ref{eq:eq:L_TGC_2_2}).\\

The unified electroweak Lagrangian discussed above involves kinetic energy terms for fermions and gauge bosons as well as interactions of fermions with gauge bosons, TGC, and QGC. However, this Lagrangian does not contain any mass terms. Because left-handed and right-handed wave functions transform differently under the electroweak symmetry, adding fermion mass terms of $\frac{1}{2} m_f^2 \bar{\psi} \psi$ would violate the Lagrangian invariance and, therefore, fermion mass terms are forbidden by the $SU(2) \times U(1)$ symmetry requirement. Mass terms for gauge bosons also would violate the Lagrangian invariance just as a photon mass term $\frac{1}{2} m^2 A^\mu A_\mu$ would violate $U(1)$ invariance of $L_{QED}$~\cite{ref_Griffiths}. Therefore, Lagrangian $L$ in Eq.~\ref{eq:L_free_covariant} contains massless particles only.\\

However, it is known from experiments that a $Z$ boson, a $W$ boson and fermions are massive particles and, therefore, our theory should accommodate their masses. To introduce masses into the electroweak Lagrangian, an $SU(2)_L$ doublet of complex scalar fields $\phi(x)$ is added to the Lagrangian:\\

\begin{equation}\label{eq:H_doublet}
  \phi(x) \equiv \begin{pmatrix} \phi^{(+)}(x) \\ \phi^{(0)}(x) \end{pmatrix}
\end{equation}

By selecting a special gauge of $\phi(x)$ it is possible to spontaneously break electroweak symmetry, generate a new scalar particle, a Higgs boson~\cite{ref_Pich}, and introduce mass terms for $W$ and $Z$ bosons and charged fermions through their couplings to the Higgs boson. The strength of the coupling constant is proportional to the square of the particle's mass, therefore, heavier particles are more likely to interact with $H$, and massless particles do not couple to $H$.\\

The mechanism of generating a fermion's mass involves both left-handed and right-handed components of the fermion. If our hypothesis that right-handed neutrinos do not exist is right, then the Higgs mechanism does not generate neutrino masses. However, from the experiments of neutrino oscillations, neutrinos are known to have masses even though they are orders of magnitude smaller than those of other fermions. Several hypotheses were offered to resolve this contradiction however at the moment the mechanism of neutrinos to acquire masses remain unknown~\cite{ref_PDG}.\\

In this dissertation, we study an electroweak process $W\gamma \rightarrow l \nu_l \gamma$, more specifically, probe TGC vertex $WW\gamma$ (Eq.~\ref{eq:L_TGC_2_2}). To do that, we are measuring a differential cross section with respect to the photon transverse momentum. The concept of the cross section in particle physics is discussed in the next chapter.\\

