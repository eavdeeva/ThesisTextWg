\subsection{Fundamental Particles and Interactions}
\label{sec:Intro_FundParticles}

\begin{figure}[htb]
  \begin{center}
    {\includegraphics[width=0.90\textwidth]{../figs/Intro/StandardModel.png}}
    \caption{Standard Model Particles and Interations}
    \label{fig:SMtable}
  \end{center}
\end{figure}

%   It is not good style to start with “Here is some stuff in the diagram”.
%Generally, this section needs serious reworking. It reads like a collection of loosely related paragraphs.
%   - The whole section reads like a list. It should read like a story. Example: a paragraph consisting of one line
%     "Protons and neutrons are baryons.”
% You can’t have that in a narrative.
%   - Seemingly reverse, or unconnected sequence, like:
%> Gluons are mediators of the strong interactions. Only quarks, antiquarks and
%> gluons can participate in the strong interactions. These particles possess a
%> special quantum number - the color charge. Quarks can be red, green or blue
%   - This paragraph to me looks like a set of totally disconnected sentences:
%> Photon is a mediator for the electromagnetic interactions. All electrically charged
%> particles participate in electromagnetic interactions. W± and Z0 bosons are mediators
%> of the weak in- teractions. All particles participate in weak interactions. W± and Z0 bosons
%> are massive while photon and gluon are massless particles.
%   - Incorrect English, like in
%> The Higgs boson is the boson which is responsible for W and Z bosons to get masses.
%  - Consider NOT starting your sections as title - figure - text, given that many of your figures are large. This comment is for this and other sections of the chapter.

There are four fundamental interactions: electromagnetic, strong, weak and gravitational. Three of them (all except gravitational) are described by the SM at the level of particles interactions. The gravity is not included into the SM but its effect on particles is negligible compared to the other forces which makes it possible to develop theory and conduct experiments of particles physics even without having the gravity included into the model. 

 The number of fundamental elementary particles in the SM is limited and they can be split into two cathegories: fermions and bosons. Fermions have spin s=1/2 and boson have spin s=0 or s=1. 

The fermions are arranged into three generations, each generation consists of a quark with charge Q=+2/3(up, charm and top quarks), a quark with Q=-1/3 (down, strange and bottom quarks), a lepton with Q=-1 (electron, muon, tau-lepton) and a neutrino which is electrically neutral. (The unit of the electric charge used is an absolute value of an electron charge). Higher generation particles have larger masses compared to the corresponding low generation particles. 

In addition to fermions, the SM include four bosons which are mediators for the SM interactions. Gluons are mediators of the strong interactions. Only quarks, antiquarks and gluons can participate in the strong interactions. These particles possess a special quantum number - the color charge. Quarks can be red, green or blue (although these are just names of the properties, not actual colors). Antiquarks can be antired, antigreen or antiblue. 

Bosons included into the SM are mediators of different interactions. A photon is a mediator for the electromagnetic interactions,  a gluon is a mediator of strong interactions, and W$^{\pm}$ and Z$^0$ bosons are mediators of the weak interactions. W$^{\pm}$ and Z$^0$ bosons are massive while photon and gluon are massless particles. 

The last SM particle is the Higgs boson which is responsible for W and Z bosons to get masses.

All the particles are summarized in Fig. \ref{fig:SMtable}. These and only these fundamental particles have been discovered by now (and their antiparticles). However, there are many composite particles which are called hadrons. Hadrons can consist of three quarks (baryons), quark and antiquark (meson), or three antiquarks (antibaryons). Hadrons always possess an integer charge.

Most of the particles are short-lived and decay into lighter particles within microseconds. The only stable particles (in terms that they do not decay) are protons and antiprotons, electrons and positrons, neutrinos and antineutrinos, photons and gluons. However, if a particle can not decay, it does not mean that it would live forever. There are many different kinds of reactions in which particles can dissapear. Antiprotons and positrons would immediately annihilate protons and electrons in the presence of a substance, photons can be absorbed by charged particles, electrons and protons can annihilate to produce neutrons and neutrinos and many other reactions are possible. 

Everyday matter consist of electrons, protons and neutrons. Electrons are fundamental particles, and protons and neutrons are baryons. Proton consists of quarks $uud$, and neutrons consists of quarks $udd$. Thus, everyday matter is also called baryon matter. 

%Protons and neutrons are baryons.

%%The electromagnetic and weak interactions are discussed in more details in Sec. \ref{sec:Intro_Electroweak} with the Higgs mechanism explained in \ref{sec:Intro_Higgs} and the strong interactions are discussed in more details in Sec. \ref{sec:Intro_QCD}. In addition to these three interaction, there is the fourth fundamental interaction known: the gravity. The gravity is not included into the SM.


