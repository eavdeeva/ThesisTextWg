\subsection{Measurements in the Past}
\label{sec:WgAbout_PastMeas}

ATGC parameters of $WW\gamma$ vertex can be probed in measurements of $W\gamma$, $WW$, $WZ$ processes. Limits on $\Delta \kappa_\gamma$ and $\lambda_\gamma$ constants obtained by different experiments are summarized in Fig.~\ref{fig:aTGC_cg}. The summary include the combination results from D0~\cite{ref_D0_aTGC_comb} ans LEP~\cite{ref_LEP_aTGC_comb} as well as results of several individual measurements by ATLAS~\cite{ref_7TeV_ATLAS},~\cite{ref_ATLAS_WW_8TeV},~\cite{ref_ATLAS_VW_8TeV} and CMS~\cite{ref_7TeV_CMS},~\cite{ref_CMS_WW_7TeV},~\cite{ref_CMS_WW_8TeV},~\cite{ref_CMS_VW_7TeV}.\\ 

\begin{figure}[htb]
  \begin{center}
    {\includegraphics[width=0.80\textwidth]{../figs/WgAbout/aTGC_cg.png}}
    \caption{Summary of limits on the $WW\gamma$ aTGC coupling constants. Figure from~\cite{ref_twiki_SMP_ATGC}.}
    \label{fig:aTGC_cg}
  \end{center}
\end{figure}

The most recent measurements of $W\gamma$ production were performed by CMS~\cite{ref_7TeV_CMS} and ATLAS~\cite{ref_7TeV_ATLAS} collaborations with $pp$ collisions at $\sqrt{s}=7$~GeV collected in~2011. Both collaborations considered two channels: $W\gamma\rightarrow\mu\nu\gamma$ and $W\gamma\rightarrow e\nu\gamma$.\\

%The measurements are based on~5~fb$^{-1}$ and~4.6~fb$^{-1}$ of integrated luminosity with CMS and ATLAS respectively.

Dibosons processes are rare in $pp$-collisions and analysts have to filter out events of their interest from many processes which are more likely to happen. To do that, variety of selection criteria is applied which reject most of background events increasing a signal fraction in the selected sample  as much as possible. However, even after all possible selection criteria are applied, majority of selected events are still background events and it is not possible to reduce the background any further without also significantly reducing signal.\\

The major source of such irreducible background is the fake photon background where hadronic jets are misidentified as photons. Such events originate from mostly $W+$jets process but $Z+$jets and $\bar{t}t+$jets events contribute to this source of the background as well. In the electron channel there is one more significant background that is the fake photon background where electron is misidentified as a photon.  Such events are coming from $Z+$jets events. For the muon channles this background is small.  Other sources of backgrounds for both channels include real-$\gamma$ backgrounds, fake lepton + real photon and fake lepton + fake photon sources.\\

Both channels provide measurements of $P_T^\gamma$ spectra because this variable is the most sensitive to the potential aTGC. The $P_T^\gamma$ spectra of the selected events in data superimposed with selected events in the simulation of the signal and estimated background contribution for the muon and electron channels are shown in Fig.~\ref{fig:Wg7TeV_CMS_ptGamma} for CMS and in Fig.~\ref{fig:Wg7TeV_ATLAS_ptGamma} for ATLAS. Both measurements show a good agreement between data and the simulation.\\

%To derive aTGC limits...

\begin{figure}[htb]
  \begin{center}
    {\includegraphics[width=0.80\textwidth]{../figs/WgAbout/Wg7TeV_CMS_ptGamma.png}}
    \caption{The distribution fo the $p_T^\gamma$ of W$\gamma$ candidates in the analysis of~7~TeV CMS data. Data vs signal MC + background estimates. Left: $W\gamma\rightarrow e\nu\gamma$, right: $W\gamma\rightarrow \mu\nu\gamma$~\cite{ref_7TeV_CMS}.}
    \label{fig:Wg7TeV_CMS_ptGamma}
  \end{center}
\end{figure}

\begin{figure}[htb]
  \begin{center}
    {\includegraphics[width=0.80\textwidth]{../figs/WgAbout/Wg7TeV_ATLAS_ptGamma.png}}
    \caption{The distribution of the photon transverse momentum (left) and missing transverse momentum (right) of W$\gamma$ candidates in the analysis of 7 TeV ATLAS data. Data vs signal MC + background estimates~\cite{ref_7TeV_ATLAS}. }
    \label{fig:Wg7TeV_ATLAS_ptGamma}
  \end{center}
\end{figure}

CMS provides measurements of the $P_T^\gamma$ spectrum, the total cross section within the phase spaces of $\Delta R>0.7$, $P_T^\gamma>15$~GeV, $P_T^\gamma>60$~GeV and $P_T^\gamma>90$~GeV, and limits on aTGC coupling constants. The phase space restrictions come from the considerations of the detector acceptance, reducing heavily background-dominated regions and theory.\\

ATLAS, in addition to the $P_T^\gamma$ spectrum, total cross section and limits, provides the differential cross section and cross section with different number of associated jets. No evidence of a new physics is observed.\\

In this dissertation we are measuring total and differential $d\sigma/d P_T^\gamma$ cross section. While the aTGC limits are not derived in this dissertation, the measured differential cross section can be used to derive them. The measurement details and results are described in Chapter~\ref{sec:AN_WgMeas}.\\
