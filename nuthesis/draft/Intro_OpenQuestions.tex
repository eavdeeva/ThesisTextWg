\subsection{Open Questions of the Standard Model}
% or Beyond the Standard Model

% Review language. Do not use “things”, “first of all”.
%  So you started your section saying that there are things that the SM does not explain. Then you listed several phenomena. Then the paragraph ends. You need to end properly, wrap up the whole introduction chapter, connect to the next chapters maybe, etc.
%This part: "The dark energy resists the gravitational attraction and accelerates the expansion of the Universe and is also not detectable by any effects except gravitational." Ok, but somehow this brief paragraph is floating in the middle of the section and is not well connected, I feel.
%Think more about this section. Consider saying that we know of some phenomena from experiments, consider mentioning how we know that dark matter or energy are likely to be real, etc.
%General: I think you need better connection between introduction chapters and connection of the chapters to your topic of dissertation. For example, the electroweak interaction is related to your thesis topic directly. The proton-proton collisions physics is also related, as well as maybe the open questions chapter.

While the Standard Model is an accurate description of all particle physics experimental results, there are certain things which are not included into the SM and it is possible that the SM is working only within certain restrictions and there is a more general extension which would explain everything.

First of all, gravitational interactions do not fit into the Standard Model. It is the open question whether the quantum theory of gravity is possible and whether there is a mediator of the gravitational interactions. Also, it is not known why the gravitational force is so much weaker than any other force. One possible explanation comes from the theory which predicts extra spatial dimensions beyond the three we are dealing with (the string theory). In this case, it is possible that the gravitational force is shared with other dimensions and that is why the fraction available in our regular three dimensions is that small.

Another mystery of the Universe is its composition: it is known from the studies of the gravitational effects that our Universe consists of dark energy by 70\%, by dark matter by 26\% and by baryon matter only by 4\%. Dark matter is a substance interacts with the baryon matter by gravitational effects only however it does not radiate and that is why it can not be detected by telescopes. The nature of the dark matter is not known but it must be something very stable to remain since the Big Bang. 
The theory of the supersymmetry which is unifying fundamental particles and mediators predicts many of new heavy particles and the lightest supersymmetric particle, the neutralino, is a good candidate for the dark matter.

The dark energy resists the gravitational attraction and accelerates the expansion of the Universe and is also not detectable by any effects except gravitational. 

One more open question is the reason for the matter/antimatter asymmetry. The matter and antimatter should have been created in the same amount. Then most of it has annihilated but because of asymmetry, there was more matter than antimatter which leads to the creation of the whole Universe. There is a phenomenon of the CP-violation in weak interactions observed and described, it predicts the asymmetry at a certain level. However, the effect of the CP-violation is not enough to account fot the observed amount of the matter and therefore the total matter/antimatter asymmetry remains unexplained. 

%Grand unification and super unification

