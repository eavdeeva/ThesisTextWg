\subsection{Compact Muon Solenoid}
\label{sec:Exp_CMS}
\subsubsection{Introduction}

CMS detector configuration
r-phi plane, r-z plane
slice in r-phi plane
subsystems
a particle traveling through the detector:
ele, pho, muon, hadron, neutrino
Where to place particle reconstruction, particle flow algorithm and MET? Check other theses
Acceptance: particles which are too collinear and go to pipe; particles which get curved too strongly

\subsubsection{Magnet}
4T inside, 2T outside, needed for track curvatures to measure the momenta
need stronger field inside to distinguish tracks better, the track density in the tracker is much higher than in the muon system
explain how the magnetic field outside the detector is created
the size, what's inside, what's outside
material

\subsubsection{Tracking System}
measures track geometry and momentum of charged particle
needs to disturb particles as little as possible: just a few measurement points to reconstruct the track
electric charge and amplification
silicon pixels, barrel and forward
silicon strips, barrel and forward
tracker alignment (here ?)
limitations

\subsubsection{Electromagnetic Calorimeter}
measures energy of electrons and photons
also determines the track, especially for photons
match to tracker: if track, it's ele/pos, if not - it's a photon
(Why muons and hadrons don't release their energy here?)
electromagnetic shower
lead tungstate crystals
how scintillator works, what the scintillation light is
photodetectors (photomultipliers?)
ECAL preshower: to distinguish between two photons coming from pi0 decay
limitations

\subsubsection{Hadron Calorimeter}
measures energy of charged and neutral hadrons
also determines the track, especially for neutral hadrons
match to tracker: if track, it's charged, if not - it's neutral
(Why muons don't release their energy here? Would photons and electrons release the energy here?)
hadronic shower
HCal Sampling calorimeter (?)
Lqyers: absorber+scintillator
Hybrid Photodiodes

\subsubsection{Muon System}
four layers of muon detectors (stations)
iron return yoke between them (how it works? why do we need it?)

>> (from cms.web.cern.ch) In total there are 1400 muon chambers: 250 drift tubes (DTs) and 540 cathode strip chambers (CSCs) track the particles’ positions and provide a trigger, while 610 resistive plate chambers (RPCs) form a redundant trigger system, which quickly decides to keep the acquired muon data or not. 

drift tubes
cathode strip chambers
resistive plate chambers

\subsubsection{Triggering and Data Aquisition}
Level-I trigger
High Level Trigger

\subsubsection{Event Reconstruction}

